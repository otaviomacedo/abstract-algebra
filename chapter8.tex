\section{Chapter 8}
\subsection{Set A}
\begin{enumerate}
    \item
        \begin{enumerate}[label=(\alph*)]
            \item $\begin{bmatrix}
                1 & 2 & 3 & 4 & 5 & 6 & 7 & 8 \\
                4 & 6 & 7 & 5 & 1 & 8 & 3 & 2
            \end{bmatrix}$

            \item $\begin{bmatrix}
                1 & 2 & 3 & 4 & 5 & 6 & 7 & 8 & 9 \\
                7 & 8 & 5 & 9 & 4 & 2 & 1 & 6 & 3
            \end{bmatrix}$

            \item $\begin{bmatrix}
                1 & 2 & 3 & 4 & 5 & 6 & 7 & 8 & 9 \\
                8 & 5 & 6 & 9 & 7 & 3 & 1 & 2 & 4
            \end{bmatrix}$

            \item $\begin{bmatrix}
                1 & 2 & 3 & 4 & 5 & 6 & 7 \\
                2 & 1 & 4 & 7 & 5 & 6 & 3
            \end{bmatrix}$

            \item $\begin{bmatrix}
                1 & 2 & 3 & 4 & 5 & 6 & 7 & 8 \\
                3 & 8 & 2 & 6 & 5 & 1 & 7 & 4 
            \end{bmatrix}$

            \item $\begin{bmatrix}
                1 & 2 & 3 & 4 & 5 & 6 & 7 & 8 & 9 \\
                3 & 5 & 4 & 9 & 2 & 1 & 7 & 6 & 8
            \end{bmatrix}$
        \end{enumerate}

        \item
            \begin{enumerate}[label=(\alph*)]
                \item $(145)(293)(67)$

                \item $(17)(24)(395)(68)$

                \item $(17435)(296)$

                \item $(1928)(375)$
            \end{enumerate}

        \item
            \begin{enumerate}[label=(\alph*)]
                \item $(18)(12)(14)(17)(13)$

                \item $(28)(25)(23)(14)(16)$

                \item $(57)(13)(12)(14)(16)$

                \item $\pi = (76)(58)(12)(14)(13)$
            \end{enumerate}

        \item
            \begin{enumerate}[label=(\alph*)]
                \item $(1,2,4)(3,7)$

                \item $(1,2,5,3,7,4)$
                
                \item $(1,2)(4,7)$
                
                \item $(1,7,3,5)$
                
                \item $(1,4,2,5,3)$
                
                \item $(1,7,4,2,3,5)$
                
                \item $(1,2,4,3,5)$
                
                \item $(1,4,2,7,5)$
            \end{enumerate}

        \item As a cycle: \\
                $(1, 2, 3, 4, 5)$\\
                $(2, 3, 4, 5, 1)$\\
                $(3, 4, 5, 1, 2)$\\
                $(4, 5, 1, 2, 3)$\\
                $(5, 1, 2, 3, 4)$\\\\ 
        As a product of transpositions:\\
        $(1, 5)(1, 4)(1, 3)(1, 2)$\\
        $(1, 5)(1, 4)(1, 3)(1, 2)(3, 4)(4, 3)$\\
        $(1, 5)(1, 4)(1, 3)(1, 2)(3, 5)(5, 3)$\\
        $(1, 5)(1, 4)(1, 3)(1, 2)(3, 2)(2, 3)$\\
        $(1, 5)(1, 4)(1, 3)(1, 2)(3, 1)(1, 3)$\\

        \item
            \begin{enumerate}[label=(\alph*)]
                \item $\alpha = (1, 2, 3)$
                
                \item $\alpha = (1, 4, 2, 5, 3)$
                
                \item $\alpha = (1, 2, 3, 4)$
            \end{enumerate}
\end{enumerate}

\subsection{Set B}
\begin{enumerate}
    \item 
        \begin{enumerate}[label=(\alph*)]
            \item 
                $\alpha^{-1} = (1,3,2)$\\
                $\alpha^2 = (1,3,2)$\\
                $\alpha^3 = ()$\\
                $\alpha^4 = (1,2,3)$\\
                $\alpha^5 = (1,3,2)$
            
            \item 
                $\alpha^{-1} = (1,4,3,2)$\\
                $\alpha^2 = (1,3)(2,4)$\\
                $\alpha^3 = (1,4,3,2)$\\
                $\alpha^4 = ()$\\
                $\alpha^5 = (1,2,3,4)$
            
            \item
                $\alpha^{-1} = (1,6,5,4,3,2)$\\
                $\alpha^2 = (1,3,5)(2,4,6)$\\
                $\alpha^3 = (1,4)(2,5)(3,6)$\\
                $\alpha^4 = (1,5,3)(2,6,4)$\\
                $\alpha^5 = (1,6,5,4,3,2)$    
        \end{enumerate}

    \item Each $\alpha^n$ corresponds to a permutation in which each element maps to the one $n$ ``hops'' ahead in the cycle. More formally, $\alpha^n(a_i) = a_{(i + n) \mod s}$. As a result there are $s$ different powers of $\alpha$.

    \item $\alpha^{-1} = (a_sa_{s - 1}\cdots a_1)$. From the equation in the previous item, $\alpha^{s - 1}(a_i) = a_{(i + s - 1) \mod s}$. So $\alpha^{s - 1}(a_1) = a_s$, $\alpha^{s - 1}(a_2) = a_1$ and so on. So $\alpha^{s - 1} = (a_sa_{s - 1}\cdots a_1) = \alpha^{-1}$.

    \item Let us suppose $s$ is odd. If we start with $a_1$ and keep applying $\alpha^2$ repeatedly, we get $a_3, a_5, \ldots, a_s$, that is, all the elements with odd index. From then on, if we continue applying $\alpha^2$, we get $a_2, a_4, \ldots, a_{s - 1}$, that is, all the elements with even index. With this procedure, we cover all the elements without repeating, thus forming a cycle. Now, if $s$ is even and we apply the same procedure, we only get odd numbers until we eventually return to $a_1$, forming a cycle that does not contain the whole domain (all the evens were left out). So, if $\alpha$ is a cycle, then $s$ is odd.

    \item First, from the equation in item 2, $\alpha^{s + 1}(a_i) = a_{(i + s + 1) \mod s} = a_{(i + 1) \mod s} = \alpha(a_i)$ for any $a_i$ in the domain. So $\alpha = \alpha^{s + 1}$. If $s$ is odd, then $(s + 1)$ is divisible by 2. So, $(\alpha^{(s + 1)/2})^2 = \alpha^{s + 1} = \alpha$.
    
    \item The reasoning here is similar to that in item 4; if $s$ is even and we apply $\alpha^2$ repeatedly starting from $a_1$, we'll generate all elements with odd index and come back to $a_1$, thus forming the cycle $(a_1a_3\ldots a_{s-1})$. Then, if we do the same, but starting from $a_2$, we'll cover all the even indices, forming the cycle $(a_2a_4\ldots a_s)$. These two cycles of length $s/2$ are disjoint and cover the whole domain, so $\alpha^2 = (a_1a_3\ldots a_{s-1})(a_2a_4\ldots a_s)$.

    \item This is a generalization of the previous item; if $s = kt$ and we apply $\alpha^k$ repeatedly starting from $a_1$, we'll generate a cycle $c_1$, composed of the sequence of elements $a_i$ with $i = kj + 1$, for $0 \leqslant j \leqslant t - 1$. Similarly, starting from $a_2$, we generate a cycle $c_2$, composed of the sequence of elements $a_i$ with $i = kj + 2$, for $0 \leqslant j \leqslant t - 1$ and so on until $a_k$. All these cycles of length $s/k$ are disjoint and cover the whole domain, so $a^k = c_1c_2\ldots c_{s/k}$.

    \item Let's start with element $a_i$ and apply $\alpha^n$ repeatedly, generating the sequence $a_i, a_{(i + n)\mod s}, a_{(i + 2n)\mod s}$ and so on. In order for this sequence to form a cycle, for all $0 < j < s$, $jn \mod s$ must be non-zero. Since $\alpha^n = \alpha$, we know that $0 < n < s$. Now, let's assume that there are $j, n$ such that $jn \mod s = 0$, which means that $jn = sk$, for some integer $k$. $s$ cannot be one of the factors of $jn$ ($s$ is greater than both $j$ and $n$). So, if $jn$ can be factored at all, $jn = sk = f_1, f_2, \ldots, k, \ldots, f_m$. Dividing both sides by $k$, we can express $s$ as a product of intergers different from 1 and itself. In other words, $s$ is not prime. Therefore, if $s$ is prime, $\alpha^n$ is a cycle for any integer $n$.
\end{enumerate}

\subsection{Set C}
\begin{enumerate}
    \item Even: (a) and (d) only.

    \item For all subitems, we can start from the same observation: the product of two permutations (each one itself written a product of transpositions) can be represented by the simple concatenation of the two. So the number of transpositions that make up the product is the sum of the number of transpositions in each of the original permutations. Then: 
        \begin{enumerate}
            \item Even plus even is even.
            \item Odd plus odd is even.
            \item Even plus odd is odd.
        \end{enumerate}

    \item Any cycle $(a_1, a_2, \ldots, a_l)$ can be represented as the composition of transpositions $(a_1,a_{l - 1})(a_1,a_{l - 2})\cdots(a_1,a_2)$. So, for a cycle of length $l$, the number of transpositions is $l - 1$. Therefore if $l$ is even, the composition is odd and vice-versa.

    \item 
    \begin{enumerate}
        \item From the previous item, we know that the $\alpha$ can be represented as a product of transpositions with size $l - 1$ and $\beta$ as a product of transpositions with size $m - 1$. And $\alpha\beta$ can be represented as a concatenation of the the these two. Therefore the size of this product is $l + m - 2$.

        \item This is a generalization of item (a).
    \end{enumerate}
\end{enumerate}

\subsection{Set D}
\begin{enumerate}
    \item This is a direct consequence of the fact that disjoint cycles commute.

    \item Let's assume that at least one of $\alpha$ and $\beta$ is not the identity. In this case, $\alpha\beta(x) \ne x$ for all $x \in \{a_1, \ldots, a_s, b_1, \ldots, b_r\}$.

    \item From item 1, $(\alpha\beta)^t = \alpha^t\beta^t$. Since $\alpha$ and $\beta$ are disjoint, so too are $\alpha^t$ and $\beta^t$. From item 2, $\alpha^t = \epsilon$ and $\beta^t = \epsilon$.

    \item $\gamma = (a_sb_1)$, so that $\alpha\beta\gamma = (a_1, \ldots, a_s, b_2, \ldots, b_r, b_1)$.

    \item $\gamma\alpha\beta = (b_2, \ldots, b_r, a_s, a_1, \ldots, a_{s - 1}, b_1)$\\$\alpha\gamma\beta = (a_1, \ldots, a_s, b_1, \ldots, b_r)$.   

    \item \textbf{TODO}
\end{enumerate}

\subsection{Set E}
\begin{enumerate}
    \item $\pi\alpha\pi^{-1}(\pi(a_1)) = \pi\alpha(a_1) = \pi(a_2)$. Analogously for $a_2, \ldots, a_s$.

    \item Let $\alpha = (a_1, \ldots, a_n)$ and $\beta = (b_1, \ldots, b_n)$. Now let's choose any permutation $\pi$ such that $\pi(a_i) = b_i$ for any element in the cycle $\alpha$. Then, we can rewrite $\beta = (\pi(a_1), \ldots, \pi(a_n))$. From the previous item, we know, then, that $\beta = \pi\alpha\pi^{-1}$ and is, therefore, a conjugate of $\alpha$. The same reasoning applies in the other direction.

    \item Let $\alpha = (a_1, \ldots, a_s)$ and $\beta = (b_1, \ldots, b_r)$. Taking any $\pi \in S_n$, $\pi\alpha\pi^{-1} = (\pi(a_1), \ldots, \pi(a_s))$ and $\pi\beta\pi^{-1} = (\pi(b_1), \ldots, \pi(b_r))$. Now suppose $\pi(a_i) = \pi(b_j)$ for some $i, j$. Since $\pi$ is a permutation, $a_i = b_j$, which is false, because $\alpha$ and $\beta$ are disjoint. Therefore $\pi\alpha\pi^{-1}$ and $\pi\beta\pi^{-1}$ are also disjoint.

    \item $\pi\sigma\pi^{-1} = \pi\alpha_1\alpha_2\cdots\alpha_t\pi^{-1} = \pi\alpha_1\pi^{-1}\pi\alpha_2\pi^{-1}\cdots\pi\alpha_t\pi^{-1}$. Each $\pi\alpha_i\pi^{-1}$ is a cycle of the same length of $\alpha_i$ and they are all disjoint (see previous item).

    \item Since $\alpha_1$ and $\alpha_2$ have the same length, it follows (from item 2) that they are conjugate, i.e., there is some permutation $\lambda$ such that $\alpha_1 = \lambda\alpha_2\lambda^{-1}$. Similarly, there is a permutation $\gamma$ such that $\beta_1 = \gamma\beta_2\gamma^{-1}$. Also from item 2, we know that we can construct $\lambda$ in such a way as to map elements in the cycle $\alpha_1$ to elements of $\alpha_2$ (elements from $\beta_1$ are mapped to themselves). Similarly, $\gamma$ maps from elements in $\beta_1$ to elements in $\beta_2$. Since $\alpha_1$ and $\beta_1$ are disjoint, we know (from item 3) that $\alpha_1\beta_1 = \lambda\alpha_2\lambda^{-1}\gamma\beta_2\gamma^{-1}$. But since $\gamma$ and $\lambda$ don't ``modify'' the same elements, we can rewrite the equation as $\alpha_1\beta_1 = \gamma\lambda\alpha_2(\gamma\lambda)^{-1}\gamma\lambda\beta_2(\gamma\lambda)^{-1}$. Making $\pi = \gamma\lambda$, we get $\alpha_1\beta_1 = \pi\alpha_2\beta_2\pi^{-1}$.
\end{enumerate}

\subsection{Set F}
\begin{enumerate}
    \item From exercise 8.B.3, $\alpha^n(a_i) = a_{(i + n) \mod s}$ for any $a_i$ in the cycle. So $\alpha^s(a_i) = a_{(i + s) \mod s} = a_i$ and, therefore, $\alpha^s = \epsilon$. And $\alpha^{2s} = \alpha^s\alpha^s = \epsilon\epsilon = \epsilon$ and $\alpha^{3s} = \alpha^{2s}\alpha^s = \epsilon\epsilon = \epsilon$. $\alpha^k \ne \epsilon$ for all $k < s$, since it will always map an element in the cycle to a different element in the cycle.

    \item This follows straighforwardly from item 1.

    \item
        \begin{enumerate}[label=(\alph*)]
            \item 6
            \item 4
            \item 20
        \end{enumerate}

    \item We need to find the smallest integer $n$ such that $(\alpha\beta)^n = \epsilon$. Since $\alpha$ and $\beta$ commute, $\alpha^n\beta^n = \epsilon$. And since they are disjoint, $\alpha^n = \epsilon$ and $\beta^n = \epsilon$. The order of $\alpha$ is 4, which means that $\alpha^{4p} = \epsilon$ for every integer $p \geqslant 1$. Similarly, the order of $\beta$ is 6, so $\beta^{6q} = \epsilon$ for every integer $q \geqslant 1$. So $n = 4p = 6q$. So the problem now has been reduced to find the least common multiple of 4 and 6, which is 12.

    \item The least common multiple of $r$ and $s$ for the same reasons explained in the previous item.
\end{enumerate}

\subsection{Set G}
\begin{enumerate}
    \item By definition, given any two different permutations $\alpha_i$ and $\alpha_j$, there is some $x$ such that $\alpha_i(x) \ne \alpha_j(x)$. Since $\beta$ is a permutation, there is some $y$ such that $\beta(y) = x$. So $\alpha_i\beta(y) \ne \alpha_j\beta(y)$ and, therefore, $\alpha_1\beta, \ldots, \alpha_r\beta$ are $r$ distinct permutations. That they are odd was already proved in exercise C2.

    \item The proof that they are distinct is the same as in the previous item. That they are even was already proved in exercise C2.

    \item Let's say the number of even permutations in $S_n$ is $r$ and the number of odd permutations is $s$. If we pick any odd permutation and mulitiply by each of the even permutations, we will get $r$ different odd permutations. So $s \geqslant r$. Similarly, if we get any odd permutation and multiply by each of the odd ones (including itself), we get $s$ different even permutations. So $r \geqslant s$. From these two inequalities, we can conclude that $r = s$, that is, the number of odd permutations is equal to the number of even permutations.

    \item From C2, we know that the composition of two even permutations is even; so $A_n$ is closed under composition. And if we reverse the order of the cycles that compose to form a permutation, we get its inverse (that is, $\alpha_1\alpha_2\ldots\alpha_n = (\alpha_n\alpha_{n-1}\ldots\alpha_1)^{-1}$, where $\alpha_i$ are cycles). So, clearly, the inverse of an even permutation is also even, that is $A_n$ is closed under inverses. Therefore $A_n$ is a subgroup of $S_n$.

    \item There are only two possiblities: either $H$ contains only even permutations (of which $A_n$ is an example) or it contains at least one odd permutation. If the former is the case, there is nothing else to prove; if the latter is the case, then we can apply the same reasoning we did in exercise G3, just replacing $S_n$ with $H$.
\end{enumerate}

\subsection{Set H}
\begin{enumerate}
    \item Every permutation can be written as a product of disjoint cycles and every cycle can be written as a product of transpositions. So every permutation can be written as a product of transpositions. In other words, the set of all transpositions in $S_n$ generates $S_n$.

    \item Any cycle $(a_1, a_2, \ldots a_n)$ can be written as the product $(1, a_1)(1, a_n)(1, a_{n-1})\ldots(1, a_1)$. Since any permutation can be written as a product of disjoint cycles, the set $\{(1, a_1), (1, a_2), \ldots (1, a_n)\}$ generates $S_n$.

    \item Every permutation can be written as a product of transpositions of the form $(1, a_i)$ (see exercise H2). Given two elements $x, y$, $(1, x)(1, y) = (1, y, x)$. So we can get any even permutation, and apply this transformation to all consecutive transpositions in the product and the result will be a product of cycles of length 3. Therefore, the set of cycles of length 3 generates $A_n$.
 
    \item \emph{The hint says it all} B-)

    \item \emph{The hint says it all} B-)
\end{enumerate}
