\section{Chapter 4}
\subsection{Set A}
\begin{enumerate}
    \item $axb = c \Rightarrow aa^{-1}xb = a^{-1}c \Rightarrow xbb^{-1} = a^{-1}cb^{-1} \Rightarrow x = a^{-1}cb^{-1}$.
    \item $x^2b = xa^{-1}c \Rightarrow x^{-1}xxb = x^{-1}xa^{-1}c \Rightarrow xb = a^{-1}c \Rightarrow xbb^{-1} = a^{-1}cb^{-1} \Rightarrow x = a^{-1}cb^{-1}$.
    \item $x^2a = bxc^{-1} \Rightarrow x^2ac = bx$. But $xac = acx$, so $xacx = bx \Rightarrow xac = b \Rightarrow x = bc^{-1}a^{-1}$.
    \item $ax^2 = b \Rightarrow ax^3 = bx$. But $x^3 = e$, so $x = b^{-1}a$.
    \item $x^2 = a^2 \Rightarrow x^4 = a^4 \Rightarrow x^5 = a^4x$. But $x^5 = e$, so $e = a^4x \Rightarrow x = (a^4)^{-1}$.
    \item $(xax)^3 = bx \Rightarrow xax^2ax^2ax = bx$. But $x^2a = a^{-1}x^{-1}$, so $xaa^{-1}x^{-1}a^{-1}x^{-1}x = bx \Rightarrow a^{-1} = bx \Rightarrow x = (ab^{-1})$.
\end{enumerate}

\subsection{Set B}
\begin{enumerate}
    \item False. $AA = I$, but $A \ne I$.
    \item False. $AA = I = II$, but $A \ne I$.
    \item False. $(AB)^2 = C^2 = I$, but $A^2B^2 = ID = D$.
    \item True. $x^2 = x \Rightarrow xxx^{-1} = xx^{-1} \Rightarrow x = e$.
    \item False. There is no $y$ such that $y^2 = A$.
    \item True. By the definition of groups, $x^{-1} \in G$. So $x^{-1}y = z \in G$ (groups are closed under the operation). Therefore $y = xz$. 
\end{enumerate}

\subsection{Set C}
\begin{enumerate}
    \item $ab = ba \Rightarrow (ab)^{-1} = (ba)^{-1} \Rightarrow b^{-1}a^{-1} = a^{-1}b^{-1}$.
    \item $a = b^{-1}ba \Rightarrow a = b^{-1}ab \Rightarrow ab^{-1} = b^{-1}a$.
    \item $a(ab) = a(ba) = (ab)a$.
    \item $a^2b^2 = aabb = abab = baba = bbaa = b^2a^2$.
    \item $(xax^{-1})(xbx^{-1}) = xabx^{-1} = xbax^{-1} = (xbx^{-1})(xax^{-1})$.
    \item 
        \begin{enumerate}
            \item $ aba^{-1} = b \Rightarrow aba^{-1}a = ba \Rightarrow ab = ba $

            \item $ ab = ba \Rightarrow aba^{-1} = baa^{-1} \Rightarrow aba^{-1} = b $
        \end{enumerate}
    \item $e = ab(ab)^{-1} = ab(ba)^{-1} = aba^{-1}b^{-1}$.
\end{enumerate}

\subsection{Set D}
\begin{enumerate}
    \item $ab = e \Rightarrow a = b^{-1} = ba = bb^{-1} = e$.
    \item $a(bc) = e \Rightarrow (bc)a = e$ (from item 1). Similarly, $(ab)c = e \Rightarrow cab = e$.
    \item If $a_1\ldots a_n = e$, then the product of all $a_i$, in any order, is equal to $e$.
    \item $xay = a^{-1} \Rightarrow xaya = e \Rightarrow yaxa = e \Rightarrow yax = e^{-1}$.
    \item $ab = c \Rightarrow abc = e \Rightarrow bca = e \Rightarrow bc = a$. Also, $cab = e \Rightarrow ca = b$.
    \item $abc = (abc)^{-1} \Rightarrow abcabc = e \Rightarrow bcabca = e \Rightarrow bca = (bca)^{-1}$. Also $cabcab = e \Rightarrow cab = (cab)^{-1}$.
    \item $abba = aea = aa = e \Rightarrow ab = (ba)^{-1}$.
    \item $a = a^{-1} \Rightarrow aa = a^{-1}a \Rightarrow aa = e$. Conversely, $aa = e \Rightarrow aaa^{-1} = ea^{-1} \Rightarrow a = a^{-1}$.
    \item $ab = c \Rightarrow abc = cc = e$. Conversely, $abc = e \Rightarrow abcc = ec \Rightarrow ab = c$.
\end{enumerate}.

\subsection{Set E}
\begin{enumerate}
    \item Let us use the Algorithm \ref{alg:s-construction} to construct $S$.
        \begin{algorithm}[H]
            \caption{Construction of $S$}\label{alg:s-construction}
            \begin{algorithmic}[1]
                \Procedure{}{}
                \State $S \leftarrow \emptyset$
                \State $G' \leftarrow$ copy of $G$
                \While{$G'$ contains at least one element which is not its own inverse} 
                    \State{Add to $S$ one such element and its inverse}
                    \State{Remove the pair from $G'$}
                \EndWhile
                \EndProcedure
            \end{algorithmic}
        \end{algorithm}
        First of all, note that at each step, the algorithm removes two elements from $G'$. Since $G'$ is finite, the algorithm is guaranteed to terminate.
        At the end of each iteration, $S$ gains two new elements. So the property that $|S|$ is even is guaranteed throughout. Finally, when the algorithm stops,
        $G'$ does not contain any element that is its own inverse. Therefore, $S$ is complete.
    \item From item 1, we know that $|S| = 2k$, for some $k \in \mathbb{N}$. The number of elements that are equal to their own inverses is $|G| - |S|$. There are, then,
        two possiblities:
        \begin{equation*}
            |G| - |S| =
            \begin{cases}
                2(m - k)     & \text{if $G = 2m$, for some $m \geqslant k$}\\ 
                2(m - k) + 1 & \text{if $G = 2m + 1$, for some $m \geqslant k$}
            \end{cases}
        \end{equation*}
        Thus if the number of elements in $G$ is even, so is the number of elements in $G$ that are equal to their own inverses. Likewise, if $G$ has an ood number of elements.
    \item If $|G|$ is even, $|G| - |S| = 2m$, for some $m \in \mathbb{N}$. But $e$ is always its own inverse. So the number of elements $x \in G$ such that $x \ne e$
        and $x = x^{-1}$ is $2n + 1$, for some $0 \leqslant n < m$. So $2n + 1 \geqslant 1$.
    \item Let $|S| = k$. Then $G = \{a_1, \ldots, a_k, a_{k+1}, \ldots, a_n\}$. $G$ being abelian, we can rewrite $(a_1\ldots a_n)^2$ as 
        $$(a_1\ldots a_n)^2 = a_1a_1^{-1}\ldots a_ka_k^{-1}a_{k+1}a_{k+1}^{-1}\ldots a_{m}a_{m}^{-1} = e$$ where $m = \frac{n - k}{2}$.
    \item $a_1\ldots a_n = a_1a_1^{-1}\ldots a_{\frac{n}{2}}^{\phantom{-1}}a_{\frac{n}{2}}^{-1} = e$.
    \item Let's say, without loss of generality, that $a_1^{\phantom{1}} \ne a_1^{-1}$. Then $a_1\ldots a_n = a_1a_2a_2^{-1}\ldots a_{\frac{n-1}{2}}^{\phantom{-1}}a_{\frac{n-1}{2}}^{-1} = a_1$.
\end{enumerate}

\subsection{Set F}
\begin{enumerate}
    \item 
        \begin{enumerate}
            \item $a^2 = a \Rightarrow aaa^{-1} = aa^{-1} \Rightarrow a = e$.
            \item $ab = a \Rightarrow a^{-1}ab = a^{-1}a \Rightarrow b = e$.
            \item $ab = b \Rightarrow abb^{-1} = bb^{-1} \Rightarrow a = e$.
        \end{enumerate}
    \item From exercise 4.B.6 we know that, for any two elements $x$ and $y$ in $G$ there is an element $z$ in $G$ such that $y = xz$. In table terms, this means
        that in each row, every element appears at least once. Now let us assume that for certain $x$, $y$ in $G$ there are $z_1$, $z_2$ in $G$ such that $y = xz_1 = xz_2$.
        Then $z_1 = z_2$. In table terms, this translates to each element in a row appearing in exactly one position.
    \item See Table \ref{tab:group-3}.
        \begin{table}[!hb]
            \centering
            \begin{tabular}{l|lll}
            &$e$&$a$&$b$\\ \hline
            $e$&$e$&$a$&$b$\\
            $a$&$a$&$b$&$e$\\
            $b$&$b$&$e$&$a$
            \end{tabular}
            \caption{Group with three elements}
            \label{tab:group-3}
        \end{table}
    \item See Table \ref{tab:group-4-self-inverses}.
        \begin{table}[!ht]
            \centering
            \begin{tabular}{l|llll}
            &$e$&$a$&$b$&$c$\\ \hline
            $e$&$e$&$a$&$b$&$c$\\
            $a$&$a$&$e$&$c$&$b$\\
            $b$&$b$&$c$&$e$&$a$\\
            $c$&$c$&$b$&$a$&$e$
            \end{tabular}
            \caption{Group with four elements such that $xx = e$ for every $x \in G$}
            \label{tab:group-4-self-inverses}
        \end{table}
    \item See Table \ref{tab:group-4-one-not-self-inverse}.
        \begin{table}[]
            \centering
            \begin{tabular}{l|llll}
            &$e$&$a$&$b$&$c$\\ \hline
            $e$&$e$&$a$&$b$&$c$\\
            $a$&$a$&$b$&$c$&$e$\\
            $b$&$b$&$c$&$e$&$a$\\
            $c$&$c$&$e$&$a$&$b$
            \end{tabular}
            \caption{Group with four elements such that $xx = e$ for some $x \ne e \in G$ and $yy \ne e$ for some $y \in G$}
            \label{tab:group-4-one-not-self-inverse}
        \end{table}
    \item \textbf{TODO}.
\end{enumerate}

\subsection{Set G}
\begin{enumerate}
    \item Prove that $G \times H$ is a group.
        \begin{enumerate}[label=(G\arabic*)]
            \item 
                $$(x_1, y_1)[(x_2y_2)(x_3y_3)] = (x_1, y_1)(x_2x_3, y_2y_3) = (x_1x_2x_3, y_1y_2y_3)$$
                $$[(x_1, y_1)(x_2y_2)](x_3y_3) = (x_1x_2, y_1y_2)(x_3, y_3) = (x_1x_2x_3, y_1y_2y_3)$$
            \item $e = (e_G, e_H)$, since $(x_1, y_1)(e_G, e_H) = (x_1, y_1) = (e_G, e_H)(x_1, y_1)$.
            \item $(x, y)^{-1} = (x^{-1}, y^{-1})$, since $(x, y)(x^{-1}, y^{-1}) = (e_G, e_H) = (x^{-1}, y^{-1})(x, y)$.
        \end{enumerate}
    \item $\mathbb{Z}_2 \times \mathbb{Z}_3 = \{(0, 0), (0, 1), (0, 2), (1, 0), (1, 1), (1, 2)\}$. See Table \ref{tab:z2xz3} for the group operation.
        \begin{table}[!hb]
            \centering
            \begin{tabular}{c|cccccc}
            $+$      & $(0, 0)$ & $(0, 1)$ & $(0, 2)$ & $(1, 0)$ & $(1, 1)$ & $(1, 2)$ \\ \hline
            $(0, 0)$ & $(0, 0)$ & $(0, 1)$ & $(0, 2)$ & $(1, 0)$ & $(1, 1)$ & $(1, 2)$ \\
            $(0, 1)$ & $(0, 1)$ & $(0, 2)$ & $(0, 0)$ & $(1, 1)$ & $(1, 2)$ & $(1, 0)$ \\
            $(0, 2)$ & $(0, 2)$ & $(0, 0)$ & $(0, 1)$ & $(0, 2)$ & $(0, 0)$ & $(1, 1)$ \\
            $(1, 0)$ & $(1, 0)$ & $(1, 1)$ & $(1, 2)$ & $(0, 0)$ & $(0, 1)$ & $(0, 2)$ \\
            $(1, 1)$ & $(1, 1)$ & $(1, 2)$ & $(1, 0)$ & $(0, 1)$ & $(0, 2)$ & $(0, 0)$ \\
            $(1, 2)$ & $(1, 2)$ & $(1, 0)$ & $(1, 1)$ & $(0, 2)$ & $(0, 0)$ & $(0, 1)$
            \end{tabular}
            \caption{Operation table for $\mathbb{Z}_2 \times \mathbb{Z}_3$}
            \label{tab:z2xz3}
        \end{table}
    \item $(g_1, h_1), (g_2, h_2) \in G \times H \Rightarrow (g_1, h_1)(g_2, h_2) = (g_1g_2, h_1h_2)$. Since $G$ and $H$ are abelian, 
        we can flip each multiplication in the tuple, resulting in $(g_2g_1, h_2h_1) =  (g1, h_1)(g_2, h_2)$.
    \item $(g, h)(g, h) = (gg, hh) = (e_G, e_H) = e_{G\times H}$.
\end{enumerate}

\subsection{Set H}
\begin{enumerate}
    \item For $n = 1$: $(bab^{-1}) = ba^{-1}b^{-1}$. Now suppose $(bab^{-1})^{n} = ba^nb^{-1}$ for some $n \geqslant 1$. Then:
        $$(bab^{-1})^{n + 1} = (bab^{-1})^{n}(bab^{-1}) = ba^{n}b^{-1}bab^{-1} = ba^nab^{-1} = ba^{n + 1}b^{-1}$$
    \item For $n = 1$: $(ab)^1 = a^1b^1$. Now suppose $(ab)^n = a^nb^n$, for some $n \geqslant 1$. Then:
        $$(ab)^{n + 1} = (ab)^n(ab) = a^nb^nab = a^nab^nb = a^{n + 1}b^{n + 1}$$
    \item For $n = 1$: $(xa)^{2.1} = xaxa = ea = a = a^1$. Now suppose $(xa)^{2n} = a^n$ for some $n \geqslant 1$. Then:
        $$(xa)^{2(n + 1)} = (xa)^{2n + 2} = (xa)^{2n}xaxa = a^nea = a^{n + 1}$$
    \item $a^3 = a^2 = e \Rightarrow a^{-1} = a^2 \Rightarrow (a^{-1})^2 = a^3a = a$. So $\sqrt{a} = a^{-1}$.
    \item $a^2 = e \Rightarrow a^2a = ea \Rightarrow a^{-1} = a^2 \Rightarrow a^3 = a$. So $\sqrt[3]{a} = a$.
    \item If there is some $x$ such that $a^{-1} = x^3$, then $a = (a^{-1})^{-1} = (x^3)^{-1} = (xxx)^{-1} = x^{-1}x^{-1}x^{-1} = (x^{-1})^3$. Therefore, $\sqrt[3]{a} = x^{-1}$.
    \item 
    \item $xax = b \Rightarrow axax = ab \Rightarrow (ax)^2 = ab \Rightarrow \sqrt{ab} = ax$.
\end{enumerate}
