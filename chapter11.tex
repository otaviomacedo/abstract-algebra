\section{Chapter 11}
\subsection{Set A}
\begin{enumerate}
    \item $ \angled{6} = \{0, 6, 12, 2, 8, 14, 4, 10\} $

    \item $ \angled{f} =  \{(), (1642), (14)(26), (1246)\} $

    \item $ \angled{\frac{1}{2}} = \{x \in \mathbb{R}^* : x = \frac{1}{2^i},  i \in \mathbb{Z} \} $; $ \angled{\frac{1}{2}} = \{x \in \mathbb{R} : x = \frac{i}{2}, i \in \mathbb{Z}\}$

    \item $ \angled{f} = \{g \in S_\mathbb{R} : g(x) = x + i, i \in \mathbb{Z}\} $

    \item $ \angled{f} = \{g \in \mathscr{F}(\mathbb{R}) : g(x) = x + i, i \in \mathbb{Z}\} $

    \item Every element $ n \in \mathbb{Z} $ can be written as $ n = 1.n = (-1)(-n) $. Let's say $ a \ne \pm 1 \in \mathbb{Z} $ is a generator of $ \mathbb{Z} $. Then, for any $ n_1, n_2 \in \mathbb{Z} $, $ n_1 = ap_1 $ and $ n_2 = ap_2 $ for some integers $ p_1, p_2 $. In this case, any two different integers would have a common divisor different from $ \pm 1 $, which is clearly not the case. So $ 1 $ and $ -1 $ are the only generators of $ \mathbb{Z} $. Every infinite cyclic group is isomorphic to $ \mathbb{Z} $. So, only $ a $ and $ a^{-1} $ are the generators of $ \angled{a} $.

    \item By Cantor's diagonal argument, there is no bijective function from $ \mathbb{R^*} $ to $ \mathbb{Z} $. Therefore $ \mathbb{R^*} $ is not cyclic.
\end{enumerate}

\subsection{Set B}
\begin{enumerate}
    \item $ G $ has an element of order $ n $. Then, by theorem 3 of chapter 10, $ G $ has $ n $ different elements.

    $ G $ has order $ n $ and is generated by $ a $. Then there exist $ n $ different powers of $ a $ and, therefore, $ \ord(a) = n$.

    \item Let $ G $ be a cyclic group generated by $ a $ and let $ b, c \in G $. Then $ b = a^i $ and $ c = a^j $ for some positive integers $ i, j $. So $ bc = a^ia^j = a^ja^i = cb $.

    \item $ b = a^k $ for some positive integer $ k $, which implies that $ \ord(b) = \ord(a^k) $. By exercise 10.D.2, the order of $ a^k $ (and therefore the order of $ b $) is a factor of the order of $ a $.

    \item If $ k $ divides $ n $ then there is some integer $ p $ such that $ n = pk $. Let $ a $ be the generator of the group. So $ \ord(a) = ord(pk) $ and, therefore, $ \ord(a^p) = k $.

    \item Let $ a, b \in G $ such that $ \ord(a) = m $ and $ \ord(b) = n $. Then $ \ord(ab) = mn $, because $ m $ and $ n $ are relatively prime and $ a $ and $ b $ commute ($ G $ is abelian). Since the order of $ G $ is $ mn $, $ ab $ is its generator and, therefore, $ G $ is cyclic.

    \item Let $ b, c \in \angled{a} $. Then $ b = a^i $  and $ c = a^j $ for some integers $ i, j $. So $ f(b) = f(c) \Rightarrow    a^{im} = a^{jm} \Rightarrow i = j \Rightarrow b = c$. So $ f $ is an injective function. Since domain and codomain are the same in this case, it's also surjective.

    In addition, $ f(a)f(b) = a^mb^m  = (ab)^m = f(ab) $. Therefore, $ f $ is an automorphism.
\end{enumerate}

\subsection{Set C}
\begin{enumerate}
    \item
    \begin{enumerate}
        \item [($\Rightarrow$)] $a^r$ is a generator of $\angled{a}$. Then $\ord{a^r} = n$. Then $r$ and $n$ are relatively prime. (see 10.G.2).
        \item [($\Leftarrow$)] $r$ and $n$ are relatively prime. Then $\ord(a^r) = n$ (see 10.G.1). Then $a^r$ is a generator of $\angled{a}$
    \end{enumerate}

    \item Every $a^i$, such that $i < n$ and $i$ is relatively prime to $n$, is a  generator of $\angled{a}$ by 11.C.1. By definition, there are $\phi(n)$ such powers of $a$. 
    
    \item $a, b \in C_m \Rightarrow (ab)^m = a^mb^m = e$ (closed under multiplication). $x \in C_m \Rightarrow (x^{-1})^m = (x^m)^{-1} = e^{-1} = e$ (closed under inverses).
    
    \item Since $m$ divides $n$, by 11.B.4, there is an element $b \in \angled{a}$ with order $m$. So $b^1, b^2, \ldots, b^m \in C_m$. 
    
    \item 
    \begin{enumerate}
        \item [($\Rightarrow$)] $\ord(x) = m \Rightarrow x^m = e \Rightarrow x \in C_m$. So, for every integer $i$, $x^{im} = e$, which means that $x^i \in C_m$. $|\angled{x}| = m$ and $C_m$ has $m$ elements. So $\angled{x} = C_m$.
        \item [($\Leftarrow$)] $x$ is a generator of $C_m$, which has $m$ elements. So $x^1, x^2, \ldots, x^{m}$ are all different. The smallest positive $i$ such that $x^i = e$ is $i = m$. In other words, $\ord(x) = m$.
    \end{enumerate}

    \item Take a generator $x$ of $C_m$. Then $\ord(x) = m$. By 11.C.2 there are $\phi(m)$ different generators of $C_m$. All of them have order $m$. Therefore there are $\phi(m)$ elements of order $m$ in $\angled{a}$.

    \item 
    \begin{enumerate}
        \item [($\Rightarrow$)] $m = \ord(a^r) = \frac{mk}{\gcd(r, mk)} \Rightarrow \gcd(r, mk) = k \Rightarrow r = kl$.
        \item [($\Leftarrow$)] $a^r = a^{kl} \Rightarrow \ord(a^r) = \frac{n}{\gcd(r, n)} = \frac{mk}{\gcd(kl, mk)} = \frac{mk}{k} = m$.
    \end{enumerate}

    \item This is essentially a restatement of 11.C.1.

\end{enumerate}

\subsection{Set D}
\begin{enumerate}
    \item $\angled{a}$ contains all powers of $b$ including $b^k = a$. Since $\angled{b}$ is a subgroup if $G$, any power of $a$ is also in $\angled{b}$. So $\angled{a} \subseteq \angled{b}$.
    \item
    \begin{enumerate}
        \item [($\Rightarrow$)] By 11.D.1, if $a$ is a power of $b$ then $\angled{a} \subseteq \angled{b}$. Likewise, if $\angled{b}$ is a power of $\angled{a}$, then $\angled{b} \subseteq \angled{a}$. Therefore $\angled{a} = \angled{b}$.
        \item [($\Leftarrow$)] $\angled{a} = \angled{b}$, so they have the same elements. Thus, $a \in \angled{b}$ and, since $\angled{b}$ is cyclic, $a = b^k$. Likewise for $b$.
    \end{enumerate}

    \item 
    \begin{enumerate}
        \item [($\Rightarrow$)] $\angled{a} = \angled{b}$. Then $|\angled{a}| = |\angled{b}| \Rightarrow \ord(a) = \ord(b)$.
        \item [($\Leftarrow$)] $\ord(a) = \ord(b) \Rightarrow |\angled{a}| = |\angled{b}|$. Since all cyclic groups of the same order are isomorphic, $\angled{a} = \angled{b}$.
    \end{enumerate}

    \item 
    \begin{enumerate}
        \item $\ord(b) = \ord(a^k) = \frac{n}{\gcd(n, k)} = n \Rightarrow \gcd(k, n) = 1$. Therefore $k$ and $n$ are relatively prime. Apply the deduction above in the reverse order as well.
    \end{enumerate}

    \item Same reasoning as the previous exercise.
    
    \item By 11.B.4, there is an element in the group, call it $a$, such that $\ord(a) = n$. The cyclic group $\angled{a}$ is a subgroup of order $n$.
\end{enumerate}

\subsection{Set E}
\begin{enumerate}
    \item If $(a, b)$ is a generator of $G \times H$, then every element of $G \times H$ can be written as $(a, b)^k = (a^k, b^k)$ for some integer $k$. Now suppose there is an element $c \in G$ such that $c = a^k$ for every integer $k$. Then there is an element $(c, x) \in G$ that is not of the form $(a^k, b^k)$, which is a contradiction. Likewise for $b$.
    
    \item Same reasoning as above. If $G \times H = \angled{(a, b)}$, then $G = \angled{a}$ and $H = \angled{b}$.
    
    \item $\mathbb{Z}_2$ and $\mathbb{Z}_4$ are cyclic but $\mathbb{Z}_2 \times \mathbb{Z}_4$ is not.
    
    \item $a^{mp} = e$ and $b^{nq} = e$ for any positive integers $p, q$. To find the order of $(a, b)$ we need to find the smallest $k$ such that $(a^k, b^k) = (e, e)$. So $k$ is the smallest integer such that $k = mp = nq$. In other words, $k = \lcm(m, n)$.
    
    \item $k = \lcm(m, n) = \frac{mn}{\gcd(m, n)} = mn$.
    
    \item $\ord(c, d) = \lcm(m, n) = \frac{mn}{\gcd(m, n)} < mn$ (because $\gcd(m, n) > 1$).
    
    \item 
    \begin{enumerate}
        \item [(i)] $\ord(a)$ and $\ord(b)$ are relatively prime. Then $\ord(a, b) = mn$. So $(a, b)$ generates $G \times H$, since $|G \times H| = mn$.
        \item [(ii)] $G \times H$ is cyclic. Then $|G \times H| = mn$ and $\ord(a) = m$ and $\ord(b) = n$. So $m$ and $n$ are relatively prime.
    \end{enumerate}
\end{enumerate}
