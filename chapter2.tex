\section{Chapter 2}
\subsection{Set A}
\begin{enumerate}
\item $a * b = \sqrt{|ab|}$ on the set $\mathbb{Q}$. This is not an operation on $\mathbb{Q}$. Square roots have two real solutions, some of them irrational. So, this operation is neither unique nor closed under $\mathbb{Q}$.

\item $a * b = a \ln b$, on the set ${x \in \mathbb{R}},\ x > 0$. This is not an operation because it's not closed. For instance, if $b = 1$ then $a\ln b = 0$, which does not belong to the set above.

\item $a * b$ is a root of the equation $x^2 - a^2b^2 = 0$, on the set $\mathbb{R}$. This is not an operation, since $a * b = \pm ab$, hence not unique.

\item Subtraction, on the set $\mathbb{Z}$. This is an operation.

\item Subtraction, on the set ${n \in \mathbb{Z} : n \geq 0}$. This is not an operation, since a subtraction of non-negative integers may result in a negative integer (not closed under the set).

\end{enumerate}

\subsection{Set B}

\begin{enumerate}
\item $x * y = x + 2y + 4$. Commutative: no; Associative: no; Identity: no; Inverses: no.
\begin{enumerate}[label=(\roman*)]
    \item $0 * 1 = 6$ and $1 * 0 = 5$.
    \item $x * (y * z) = x + 2y + 4z + 12$. $(x * y) * z = x + 2y + 2z + 4$.
    \item $x * e = x \Rightarrow x + 2e + 4 = x \Rightarrow 2e + 4 = 0 \Rightarrow e = -2$. But this value of $e$ does not satisfy the equation $e * y = y$, since $-2 * y = -2 + 2y + 4 = 2y + 2 \ne y$.
    \item No identity implies no inverses.
\end{enumerate}

\item $x * y = x + 2y - xy$. Commutative: no; Associative: no; Identity: no; Inverses: no.
\begin{enumerate}[label=(\roman*)]
    \item $0 * 1 = 2$ and $1 * 0 = 1$.
    \item $(x * y) * z = 2y - xy - 2z - 2yz + xyz$ and $x * (y * z) = x + 2y + 4z - 2yz - xy - 2xz + xyz$.
    \item $x * e = x \Rightarrow x + 2e - xe = x \Rightarrow 2e - xe = 0 \Rightarrow e = 0$. But this value of $e$ does not satisfy the equation $e * y = y$, since $0 * y = 2y \ne y$.
    \item No identity implies no inverses.
\end{enumerate}

\item $x * y = |x + y|$.  Commutative: yes; Associative: no; Identity: yes; Inverses: yes.
\begin{enumerate}[label=(\roman*)]
    \item $|x + y| = |y + x|$.
    \item $||1 + -3| + -5| = 3$. But $|1 + |-3 + -5|| = 9$.
    \item $x * e = x \Rightarrow |x + e| = x \Rightarrow e = 0$. Being commutative, $x * e = e * x$. So $0$ is the identity.
    \item $x * x' = 0 \Rightarrow |x + x'| = 0 \Rightarrow x' = -x$. So, the inverse of $x$ is $-x$.
\end{enumerate}

\item $x * y = |x - y|$.   Commutative: yes; Associative: no; Identity: yes; Inverses: yes.
\begin{enumerate}[label=(\roman*)]
    \item $x - y = -(y - x) \Rightarrow |x - y| = |-(y - x)| = |y - x|$.
    \item $||1 - 3| - 5| = 3$. But $|1 - |3 - 5|| = 1$.
    \item $x * e = x \Rightarrow |x - e| = x \Rightarrow e = 0$. Being commutative, $x * e = e * x$. So $0$ is the identity.
    \item $x * x' = 0 \Rightarrow |x - x'| = 0 \Rightarrow x' = x$. So every element is its own inverse.
\end{enumerate}

\item $x * y = xy + 1$. Commutative: yes; Associative: no; Identity: no; Inverses: no.
\begin{enumerate}[label=(\roman*)]
    \item $xy + 1 = yx + 1$.
    \item $(x * y) * z = xyz + z + 1$. But $x * (y * z) = xyz + x + 1$.
    \item $x * e = x \Rightarrow xe + 1 = x$, which does not have a real solution.
    \item No identity implies no inverses.
\end{enumerate}

\item $x * y = \max\ \{x, y\}$. Commutative: yes; Associative: yes; Identity: no; Inverses: no.
\begin{enumerate}[label=(\roman*)]
    \item $\max\ \{x, y\} = \max\ \{y, x\}$.
    \item $\max\ \{x, \max\ \{y, z\}\} = \max\ \{\max\ \{x, y\}, z\}$.
    \item $x * e = x$ would imply that there exists an $e$ that is smaller than any $x \in \mathbb{R}$, which is false.
    \item No identity implies no inverses.
\end{enumerate}
\end{enumerate}

\subsection{Set C}
Table \ref{tab:binary-operations} lists all the operations for the set $\{a, b\}$. 

\begin{table}[]
    \centering         
    \begin{tabular}{cc|cccccccccccccccc}
    $x$ & $y$ & $O_1$ & $O_2$ & $O_3$ & $O_4$ & $O_5$ & $O_6$ & $O_7$ & $O_8$ & $O_9$ & $O_{10}$ & $O_{11}$ & $O_{12}$ & $O_{13}$ & $O_{14}$ & $O_{15}$ & $O_{16}$ \\
    $a$ & $a$ & $a$     & $a$     & $a$     & $a$     & $a$     & $a$     & $a$     & $a$     & $b$     & $b$      & $b$      & $b$      & $b$      & $b$      & $b$      & $b$      \\
    $a$ & $b$ & $a$     & $a$     & $a$     & $a$     & $b$     & $b$     & $b$     & $b$     & $a$     & $a$      & $a$      & $a$      & $b$      & $b$      & $b$      & $b$      \\
    $b$ & $a$ & $a$     & $a$     & $b$     & $b$     & $a$     & $a$     & $b$     & $b$     & $a$     & $a$      & $b$      & $b$      & $a$      & $a$      & $b$      & $b$      \\
    $b$ & $b$ & $a$     & $b$     & $a$     & $b$     & $a$     & $b$     & $a$     & $b$     & $a$     & $b$      & $a$      & $b$      & $a$      & $b$      & $a$      & $b$     
    \end{tabular}
    \caption{Operations on $\{a, b\}$}
    \label{tab:binary-operations}
\end{table}

\begin{enumerate}
    \item Commutative: $\{O_1, O_2, O_7, O_8, O_9, O_{10}, O_{15}, O_{16}\}$.
    \item Associative: $\{O_1, O_2, O_4, O_6, O_7, O_8, O_{10}, O_{16}\}$.
    \item Identity: $\{O_2, O_7, O_8, O_{10}\}$.
    \item Inverses: $\{O_7, O_{10}\}$.
\end{enumerate}

\subsection{Set D}
\begin{enumerate}
    \item Let $a, b, c \in A^*$. Then:
        $$(ab)c = (a_1\ldots a_mb_1\ldots b_n)c_1\ldots c_p = a_1\ldots a_m(b_1\ldots b_nc_1\ldots c_p) = a(bc)$$
    \item Let $A = \{0, 1\}$ and $a = 001$ and $b = 110$, $a, b \in A^*$. Then $ab = 001110$ and $ba = 110001$, clearly showing that $ab \ne ba$.
    \item Let $a\lambda = \lambda a = a$. So $\lambda$ is the identity for this operation.
\end{enumerate}
