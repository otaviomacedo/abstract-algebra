\documentclass{article}
\usepackage[margin=0.5in]{geometry}
\usepackage[utf8]{inputenc}
% \usepackage{euler}
\usepackage{amsfonts}
\usepackage{amsmath}
\usepackage{amssymb}
\usepackage{enumitem}
\usepackage{float}
\usepackage{algorithm}
\usepackage{ mathrsfs }
\usepackage[noend]{algpseudocode}

\let\emptyset\varnothing
\DeclareMathOperator{\dec}{dec}


\author{Otavio Macedo}
\title{A Book of Abtract Algebra - solutions to exercises}


\begin{document}

\maketitle

\section*{Chapter 2}
\subsection*{Set A}
\begin{enumerate}
\item $a * b = \sqrt{|ab|}$ on the set $\mathbb{Q}$. This is not an operation on $\mathbb{Q}$. Square roots have two real solutions, some of them irrational. So, this operation is neither unique nor closed under $\mathbb{Q}$.

\item $a * b = a \ln b$, on the set ${x \in \mathbb{R}},\ x > 0$. This is not an operation because it's not closed. For instance, if $b = 1$ then $a\ln b = 0$, which does not belong to the set above.

\item $a * b$ is a root of the equation $x^2 - a^2b^2 = 0$, on the set $\mathbb{R}$. This is not an operation, since $a * b = \pm ab$, hence not unique.

\item Subtraction, on the set $\mathbb{Z}$. This is an operation.

\item Subtraction, on the set ${n \in \mathbb{Z} : n \geq 0}$. This is not an operation, since a subtraction of non-negative integers may result in a negative integer (not closed under the set).

\end{enumerate}

\subsection*{Set B}

\begin{enumerate}
\item $x * y = x + 2y + 4$. Commutative: no; Associative: no; Identity: no; Inverses: no.
\begin{enumerate}[label=(\roman*)]
    \item $0 * 1 = 6$ and $1 * 0 = 5$.
    \item $x * (y * z) = x + 2y + 4z + 12$. $(x * y) * z = x + 2y + 2z + 4$.
    \item $x * e = x \Rightarrow x + 2e + 4 = x \Rightarrow 2e + 4 = 0 \Rightarrow e = -2$. But this value of $e$ does not satisfy the equation $e * y = y$, since $-2 * y = -2 + 2y + 4 = 2y + 2 \ne y$.
    \item No identity implies no inverses.
\end{enumerate}

\item $x * y = x + 2y - xy$. Commutative: no; Associative: no; Identity: no; Inverses: no.
\begin{enumerate}[label=(\roman*)]
    \item $0 * 1 = 2$ and $1 * 0 = 1$.
    \item $(x * y) * z = 2y - xy - 2z - 2yz + xyz$ and $x * (y * z) = x + 2y + 4z - 2yz - xy - 2xz + xyz$.
    \item $x * e = x \Rightarrow x + 2e - xe = x \Rightarrow 2e - xe = 0 \Rightarrow e = 0$. But this value of $e$ does not satisfy the equation $e * y = y$, since $0 * y = 2y \ne y$.
    \item No identity implies no inverses.
\end{enumerate}

\item $x * y = |x + y|$.  Commutative: yes; Associative: no; Identity: yes; Inverses: yes.
\begin{enumerate}[label=(\roman*)]
    \item $|x + y| = |y + x|$.
    \item $||1 + -3| + -5| = 3$. But $|1 + |-3 + -5|| = 9$.
    \item $x * e = x \Rightarrow |x + e| = x \Rightarrow e = 0$. Being commutative, $x * e = e * x$. So $0$ is the identity.
    \item $x * x' = 0 \Rightarrow |x + x'| = 0 \Rightarrow x' = -x$. So, the inverse of $x$ is $-x$.
\end{enumerate}

\item $x * y = |x - y|$.   Commutative: yes; Associative: no; Identity: yes; Inverses: yes.
\begin{enumerate}[label=(\roman*)]
    \item $x - y = -(y - x) \Rightarrow |x - y| = |-(y - x)| = |y - x|$.
    \item $||1 - 3| - 5| = 3$. But $|1 - |3 - 5|| = 1$.
    \item $x * e = x \Rightarrow |x - e| = x \Rightarrow e = 0$. Being commutative, $x * e = e * x$. So $0$ is the identity.
    \item $x * x' = 0 \Rightarrow |x - x'| = 0 \Rightarrow x' = x$. So every element is its own inverse.
\end{enumerate}

\item $x * y = xy + 1$. Commutative: yes; Associative: no; Identity: no; Inverses: no.
\begin{enumerate}[label=(\roman*)]
    \item $xy + 1 = yx + 1$.
    \item $(x * y) * z = xyz + z + 1$. But $x * (y * z) = xyz + x + 1$.
    \item $x * e = x \Rightarrow xe + 1 = x$, which does not have a real solution.
    \item No identity implies no inverses.
\end{enumerate}

\item $x * y = \max\ \{x, y\}$. Commutative: yes; Associative: yes; Identity: no; Inverses: no.
\begin{enumerate}[label=(\roman*)]
    \item $\max\ \{x, y\} = \max\ \{y, x\}$.
    \item $\max\ \{x, \max\ \{y, z\}\} = \max\ \{\max\ \{x, y\}, z\}$.
    \item $x * e = x$ would imply that there exists an $e$ that is smaller than any $x \in \mathbb{R}$, which is false.
    \item No identity implies no inverses.
\end{enumerate}
\end{enumerate}

\subsection*{Set C}
Table \ref{tab:binary-operations} lists all the operations for the set $\{a, b\}$. 

\begin{table}[]
    \centering         
    \begin{tabular}{cc|cccccccccccccccc}
    $x$ & $y$ & $O_1$ & $O_2$ & $O_3$ & $O_4$ & $O_5$ & $O_6$ & $O_7$ & $O_8$ & $O_9$ & $O_{10}$ & $O_{11}$ & $O_{12}$ & $O_{13}$ & $O_{14}$ & $O_{15}$ & $O_{16}$ \\
    $a$ & $a$ & $a$     & $a$     & $a$     & $a$     & $a$     & $a$     & $a$     & $a$     & $b$     & $b$      & $b$      & $b$      & $b$      & $b$      & $b$      & $b$      \\
    $a$ & $b$ & $a$     & $a$     & $a$     & $a$     & $b$     & $b$     & $b$     & $b$     & $a$     & $a$      & $a$      & $a$      & $b$      & $b$      & $b$      & $b$      \\
    $b$ & $a$ & $a$     & $a$     & $b$     & $b$     & $a$     & $a$     & $b$     & $b$     & $a$     & $a$      & $b$      & $b$      & $a$      & $a$      & $b$      & $b$      \\
    $b$ & $b$ & $a$     & $b$     & $a$     & $b$     & $a$     & $b$     & $a$     & $b$     & $a$     & $b$      & $a$      & $b$      & $a$      & $b$      & $a$      & $b$     
    \end{tabular}
    \caption{Operations on $\{a, b\}$}
    \label{tab:binary-operations}
\end{table}

\begin{enumerate}
    \item Commutative: $\{O_1, O_2, O_7, O_8, O_9, O_{10}, O_{15}, O_{16}\}$.
    \item Associative: $\{O_1, O_2, O_4, O_6, O_7, O_8, O_{10}, O_{16}\}$.
    \item Identity: $\{O_2, O_7, O_8, O_{10}\}$.
    \item Inverses: $\{O_7, O_{10}\}$.
\end{enumerate}

\subsection*{Set D}
\begin{enumerate}
    \item Let $a, b, c \in A^*$. Then:
        $$(ab)c = (a_1\ldots a_mb_1\ldots b_n)c_1\ldots c_p = a_1\ldots a_m(b_1\ldots b_nc_1\ldots c_p) = a(bc)$$
    \item Let $A = \{0, 1\}$ and $a = 001$ and $b = 110$, $a, b \in A^*$. Then $ab = 001110$ and $ba = 110001$, clearly showing that $ab \ne ba$.
    \item Let $a\lambda = \lambda a = a$. So $\lambda$ is the identity for this operation.
\end{enumerate}

\section*{Chapter 3}

\subsection*{Set A}

\begin{enumerate}
    \item $x * y = x + y + k$. Same thing as the example in Set B of Chapter 2, but with a generic constant $k$ instead of the fixed constant 1.
    \item $x * y = \frac{xy}{2}$, on the set $\{x \in \mathbb{R}, x \ne 0\}$.
    \begin{description}
        \item [Commutative] 
            $$\frac{xy}{2} = \frac{yx}{2}$$
        \item [Associative]
        $$(x * y) * z = \frac{xy}{2} * z = \frac{\frac{xy}{2}z}{2} = \frac{xyz}{4}$$
        $$x * (y * z) = \frac{x(y * z)}{2} = \frac{x\frac{yz}{2}}{2} = \frac{xyz}{4}$$
        \item [Identity] $x * e = x \Rightarrow \frac{xe}{2} = x \Rightarrow e = 2$
        \item [Inverse] $x * x' = 2 \Rightarrow \frac{xx'}{2} = 2 \Rightarrow xx' = 4 \Rightarrow x' = \frac{4}{x}$
    \end{description}
    \item $x * y = x + y + xy$, on the set $\{x \in \mathbb{R}, x \ne -1\}$
    \begin{description}
        \item [Commutative] $x + y + xy = y + x + yx$
        \item [Associative]
        $$(x * y) * z = (x + y + xy) * z = (x + y + xy) + x + (x + y + xy)z = x + y + z + xy + yz + xyz$$
        $$x * (y * z) = x * (y + z + yz) = x + (y + z + yz) + x(y + z + yz) = x + y + z + xy + yz + xyz$$
        \item [Identity] $x * e = x \Rightarrow x + e + xe = x \Rightarrow x + e + xe - x = 0 \Rightarrow x(1 + e - 1) + e = 0 \Rightarrow xe + e = 0 \Rightarrow e = 0$
        \item [Inverse] $x * x' = 0 \Rightarrow x + x'+ xx' = 0 \Rightarrow x = -x'(1 + x) \Rightarrow x' = \frac{x}{1+x}$
    \end{description}
    \item $x * y = \frac{x + y}{xy + 1}$, on the set $\{x \in \mathbb{R}, -1 < x < 1\}$.
    \begin{description}
        \item [Commutative]
            $$\frac{x + y}{xy + 1} = \frac{y + x}{yx + 1}$$
        \item [Associative]
            $$(x * y) * z = \frac{x + y}{xy + 1} * z = \frac{\left(\frac{x + y}{xy + 1}\right) + z}{\left(\frac{x + y}{xy + 1}\right)z + 1} = \frac{x + y + xyz + z}{xz + yz + xy + 1}$$
            $$x * (y * z) = x * \frac{y + z}{yz + 1} = \frac{x + \left(\frac{y + z}{yz + 1}\right)}{x\left(\frac{y + z}{yz + 1}\right) + 1} = \frac{x + y + xyz + z}{xz + yz + xy + 1}$$
        \item [Identity] $e * x = x * e = x \Rightarrow \frac{x + e}{xe + 1} = x \Rightarrow x + e = x(xe + 1) \Rightarrow e = 0$
        \item [Inverse] $x' * x = x * x' = 0 \Rightarrow \frac{x + x'}{xx' + 1} = 0 \Rightarrow x + x' = 0 \Rightarrow x' = -x$
    \end{description}
\end{enumerate}

\subsection*{Set B}
\begin{enumerate}
    \item $(a, b) * (c , d) = (ad + bc, bd)$, on the set $\{(x, y) \in \mathbb{R} \times \mathbb{R}: y \ne 0\}$: abelian group.
        \begin{description}
            \item [Commutative: Yes] $(ad + bc, bd) = (cb + da, bd)$
            \item [Associative: Yes]
                $$[(a, b) * (c, d)] * (f, g) = (ad + bc, bd) * (f, g) = (adg + bcg + bdf, bdg)$$
                $$(a, b) * [(c, d) * (f, g)] = (a, b) * (cg + df, dg) = (adg + bcg + bdf, bdg)$$
            \item [Identity: Yes]
                \begin{equation*}
                    \begin{split}
                        (e_1, e_2) * (a, b) = (a, b) * (e_1, e_2) = (a, b) & \Rightarrow (ae_2 + be_1, be_2) = (a, b) \\
                                                                           & \Rightarrow  \begin{cases}
                                                                                                be_2 = b \Rightarrow e_2 = 1 \\
                                                                                                ae_2 + be_1 = a \Rightarrow be_1 = 0 \Rightarrow e_1 = 0
                                                                                          \end{cases} \\
                                                                           & \Rightarrow e = (0, 1)
                    \end{split}
                \end{equation*}
            \item [Inverse: Yes]
                \begin{equation*}
                    \begin{split}
                        (a', b') * (a, b) = (a, b) * (a', b') = (0, 1) & \Rightarrow (ab' + ba', bb') = (0, 1) \\
                                                                       & \Rightarrow \begin{cases}
                                                                                            bb' = 1 \Rightarrow b' = \frac{1}{b} \\
                                                                                            ab' + ba' = 0 \Rightarrow \frac{a}{b} + ba' = 0 \Rightarrow a' = -\frac{a}{b^2}
                                                                                     \end{cases} \\
                                                                       & \Rightarrow (a, b)' = \left(-\frac{a}{b^2}, \frac{1}{b}\right)
                    \end{split}
                \end{equation*}
        \end{description}
    \item $(a, b) * (c, d) = (ac, bc + d)$, on the set $\{(x, y) \in \mathbb{R} \times \mathbb{R}: x \ne 0\}$: non-abelian group.
        \begin{description}
            \item [Commutative: No] $(ac, bc + d) \ne (ca, da + b)$
            \item [Associative: Yes]
                $$[(a, b) * (c, d)] * (f, g) = (ac, bc + d) * (f, g) = (acf, bcf + df + g)$$
                $$[(a, b) * [(c, d) * (f, g)] = (a, b) * (cf, df + g) = (acf, bcf + df + g)$$
            \item [Identity: Yes]
                \begin{equation*}
                    \begin{split}
                        (a, b) * (e_1, e_2) = (a, b) & \Rightarrow (ae_1 + be_1, e_2) = (a, b) \\
                                                    & \Rightarrow  \begin{cases}
                                                                        ae_1 = a \Rightarrow e_1 = 1 \\
                                                                        be_1 + e2 = b \Rightarrow b + e_2 = b \Rightarrow e_2 = 0
                                                                   \end{cases} \\
                                                    & \Rightarrow e = (1, 0)
                    \end{split}
                \end{equation*}
                Not being commutative, we have to check the inverse order of the operands:
                    $$(1, 0) * (a, b) = (1a + 0a, b) = (a, b)$$
            \item [Inverse: Yes]
                \begin{equation*}
                    \begin{split}
                        (a, b) * (a', b') = (1, 0) & \Rightarrow (aa' + ba', b') = (1, 0) \\
                                                & \Rightarrow \begin{cases}
                                                                    aa' = 1 \Rightarrow a' = \frac{1}{a} \\
                                                                    ba' + b' = 0 \Rightarrow \frac{b}{a} + b' = 0 \Rightarrow b' = -\frac{b}{a}
                                                                \end{cases}
                    \end{split}
                \end{equation*}
                Not being commutative, we have to check the inverse order of the operands:
                    $$\left(\frac{1}{a}, -\frac{b}{a}\right) * (a, b) = \left(\frac{1}{a}a, -\frac{b}{a}a + b\right) = (1, 0)$$
        \end{description}
    \item Same operation as in part 2, but on the set $\mathbb{R} \times \mathbb{R}$: not a group. There is no solution for the identity element.
    \item $(a, b) * (c, d = (ac -bd, ad + bc)$, on the set $\mathbb{R} \times \mathbb{R}$, with the origin deleted: abelian group.
        \begin{description}
            \item [Commutative: Yes] $(ac - bd, ad + bc) = (ca - db, cb + da)$
            \item [Associative: Yes]
                $$[(a, b) * (c, d)] * (f, g) = (ac - bd, ad + bc) * (f, g) = (acf - bdf - adg - bcg, acg - bdg + adf + bcf)$$
                $$(a, b) * [(c, d) * (f, g)] = (a, b) * (cf - dg, cg + df) = (acf - adg - bcg - bdf, acg + adf + bcf - bdg)$$
            \item [Identity: Yes]
                \begin{equation*}
                    \begin{split}
                        (e_1, e_2) * (a, b) = (a, b) * (e_1, e_2) = (a, b) & \Rightarrow (ae_1 - be_2, ae_2 + be_1) = (a, b) \\
                                                                           & \Rightarrow  \begin{cases}
                                                                                ae_1 - be_2 = a \Rightarrow e_1 = \frac{a + be_2}{a} \Rightarrow e_1 = 1 \\
                                                                                be_2 + be_1 = b \Rightarrow ae_2 + b\left(\frac{a + be_2}{a}\right) = b \Rightarrow e_2 = 0
                                                                             \end{cases} \\
                                                                           & \Rightarrow e = (1, 0)
                    \end{split}
                \end{equation*}
            \item [Inverses: Yes]
            \begin{equation*}
                \begin{split}
                    (a', b') * (a, b) = (a, b) * (a', b') = (1, 0) & \Rightarrow (aa' - bb', ab' + ba') = (1, 0) \\
                                                                   & \Rightarrow \begin{cases}
                                                                                        ab' + ba' = 0 \Rightarrow b' = -\frac{ba'}{a} \Rightarrow b'= -\frac{ba}{a^3 + ab^2}\\
                                                                                        aa' - bb' = 1 \Rightarrow aa' + \frac{b^2a'}{a} = 1 \Rightarrow a'= \frac{a}{a^2 + b^2}
                                                                                 \end{cases}\\
                                                                   & \Rightarrow (a, b)' = \left(\frac{a}{a^2 + b^2}, -\frac{ba}{a^3 + ab^2}\right)
                \end{split}
            \end{equation*}
    \end{description}
    \item Consider the operation of the preceding problem on the set $\mathbb{R} \times \mathbb{R}$. Is this a group? Explain.\\
    This is not a group. The value for the identity is undefined.
\end{enumerate}

\subsection*{Set C}
\begin{enumerate}
    \item $e = \emptyset$, since $\emptyset + A = A + \emptyset = (A - \emptyset) \cup (\emptyset - A) = A \cup A = A$.
    \item $A' + A = A + A' = \emptyset \Rightarrow (A - A') \cup (A'- A) = \emptyset \cup \emptyset = \emptyset$.
    \item $P_D = \{\emptyset, \{a\}, \{b\}, \{c\}, \{a, b\}, \{a, c\}, \{b, c\}, \{a, b, c\}\}$. See table \ref{tab:powerset-op}.
    \begin{table}[!ht]
        \centering
        \begin{tabular}{c|cccccccc}
        +             & $\emptyset$   & $\{a\}$       & $\{b\}$       & $\{c\}$       & $\{a, b\}$    & $\{a, c\}$    & $\{b, c\}$    & $D$ \\
        \hline
        $\emptyset$   & $\emptyset$   & $\{a\}$       & $\{b\}$       & $\{c\}$       & $\{a, b\}$    & $\{a, c\}$    & $\{b, c\}$    & $D$ \\
        $\{a\}$       & $\{a\}$       & $\emptyset$   & $\{a, b\}$    & $\{a, c\}$    & $\{b\}$       & $\{c\}$       & $D$           & $\{b, c\}$    \\
        $\{b\}$       & $\{b\}$       & $\{a, b\}$    & $\emptyset$   & $\{b, c\}$    & $\{a\}$       & $D$           & $\{c\}$       & $\{a, c\}$    \\
        $\{c\}$       & $\{c\}$       & $\{a, c\}$    & $\{b, c\}$    & $\emptyset$   & $D$           & $\{a\}$       & $\{b\}$       & $\{a, b\}$    \\
        $\{a, b\}$    & $\{a, b\}$    & $\{b\}$       & $\{a\}$       & $D$           & $\emptyset$   & $\{b, c\}$    & $\{a, c\}$    & $\{c\}$       \\
        $\{a, c\}$    & $\{a, c\}$    & $\{c\}$       & $D$           & $\{a\}$       & $\{b, c\}$    & $\emptyset$   & $\{a, b\}$    & $\{b\}$       \\
        $\{b, c\}$    & $\{b, c\}$    & $D$           & $\{c\}$       & $\{b\}$       & $\{a, c\}$    & $\{a, b\}$    & $\emptyset$   & $\{a\}$       \\
        $D$           & $D$           & $\{b, c\}$    & $\{a, c\}$    & $\{a, b\}$    & $\{c\}$       & $\{b\}$       & $\{a\}$       & $\emptyset$  
        \end{tabular}
        \caption{Operation table for $\langle P_D, + \rangle$}
        \label{tab:powerset-op}
    \end{table}
\end{enumerate}

\subsection*{Set D}
See Table \ref{tab:checkerboard} for the checkerboard game operation table. $I$ is the identity since $X * I = I * X= X$ for any $X \in G$, and every element has an inverse (itself).
\begin{table}[H]
    \centering
    \begin{tabular}{c|cccc}
    $*$ & $I$ & $V$ & $H$ & $D$ \\ \hline
    $I$ & $I$ & $V$ & $H$ & $D$ \\
    $V$ & $V$ & $I$ & $D$ & $H$ \\
    $H$ & $H$ & $D$ & $I$ & $V$ \\
    $D$ & $D$ & $H$ & $V$ & $I$
    \end{tabular}
    \caption{Operation table for $\langle G, *\rangle$}
    \label{tab:checkerboard}
\end{table}

\subsection*{Set E}
See Table \ref{tab:coingame-op} for the coin game operation table. $I$ is the identity, since $X * I = I * X = X$, for every $X \in G$. 
In every line there is an entry with $I$, which means that every element has an inverse.
$\langle G, *\rangle$ is not commutative. For instance: $M_2 * M_4 \ne M_4 * M_2$.

\begin{table}[!hb]
    \centering
    \begin{tabular}{c|cccccccc}
    $*$ & $I$ & $M_1$ & $M_2$ & $M_3$ & $M_4$ & $M_5$ & $M_6$ & $M_7$ \\ \hline
    $I$ & $I$ & $M_1$ & $M_2$ & $M_3$ & $M_4$ & $M_5$ & $M_6$ & $M_7$ \\
    $M_1$ & $M_1$ & $I$ & $M_3$ & $M_2$ & $M_5$ & $M_4$ & $M_7$ & $M_6$ \\
    $M_2$ & $M_2$ & $M_3$ & $I$ & $M_1$ & $M_6$ & $M_7$ & $M_4$ & $M_5$ \\
    $M_3$ & $M_3$ & $M_2$ & $M_1$ & $I$ & $M_7$ & $M_6$ & $M_5$ & $M_4$ \\
    $M_4$ & $M_4$ & $M_6$ & $M_5$ & $M_7$ & $I$ & $M_2$ & $M_1$ & $M_3$ \\
    $M_5$ & $M_5$ & $M_7$ & $M_4$ & $M_6$ & $M_1$ & $M_3$ & $I$ & $M_2$ \\
    $M_6$ & $M_6$ & $M_4$ & $M_7$ & $M_5$ & $M_2$ & $I$ & $M_3$ & $M_1$ \\
    $M_7$ & $M_7$ & $M_5$ & $M_6$ & $M_4$ & $M_3$ & $M_1$ & $M_2$ & $I$
    \end{tabular}
    \caption{Operation table for $\langle G, *\rangle$}
    \label{tab:coingame-op}
\end{table}

\subsection*{Set F}
\begin{enumerate}
    \item \begin{equation*}
            \begin{split}
                (a_1, a_2, \ldots, a_n) + (b_1, b_2, \ldots, b_n) & = (a_1 + b_1, a_2 + b_2, \ldots, a_n + b_n) \\
                                                                   & = (b_1 + a_1, b_2 + a_2, \ldots, b_n + a_n) \\
                                                                  & = (b_1, b_2, \ldots, b_n) + (a_1, a_2, \ldots, a_n)
            \end{split}
        \end{equation*}
    \item 
        $$1 + (0 + 1) = 1 + 1 = 0 = 1 + 1 = (1 + 0) + 1$$
        $$1 + (0 + 0) = 1 + 0 = 0 = 1 + 0 = (1 + 0) + 0$$
        $$0 + (1 + 1) = 0 + 0 = 0 = 1 + 1 = (0 + 1) + 1$$
        $$0 + (0 + 1) = 0 + 1 = 1 = 0 + 1 = (0 + 0) + 1$$
        $$0 + (1 + 0) = 0 + 1 = 1 = 0 + 0 = (0 + 1) + 0$$
        $$0 + (0 + 0) = 0 + 0 = 0 = 0 + 0 = (0 + 0) + 1$$
    \item
        \begin{equation*}
            \begin{split}
                (a_1, \ldots, a_n) + [(b_1, \ldots, b_n) + (c_1, \ldots, c_n)] & = (a_1, \ldots, a_n) + (b_1 + c_1, \ldots, b_n + c_n) \\
                                                                               & = (a_1 + (b_1 + c_1), \ldots, a_n + (b_n + c_n)) \\
                                                                               & = ((a_1 + b_1) + c_1, \ldots, (a_n + b_n) + c_n) \\
                                                                               & = [(a_1, \ldots, a_n) + (b_1, \ldots, b_n)] + (c_1, \ldots, c_n)
            \end{split}
        \end{equation*}
    \item The identity is $(0_1, \ldots, 0_n)$, since $(a_1, \ldots, a_n) + (0_1, \ldots, 0_n) = (a_1, \ldots, a_n) = (0_1, \ldots, 0_n) + (a_1, \ldots, a_n)$.
    \item $(a_1, \ldots, a_n)$ is its own inverse, since $(a_1, \ldots, a_n) + (a_1, \ldots, a_n) = (a_1 + a_1, \ldots, a_n + a_n) = (0_1, \ldots, 0_n)$.
    \item $b = -b \Rightarrow a + b = a + (-b) \Rightarrow a + b = a - b$.
    \item $a + b = c \Rightarrow a + b - b = c - b \Rightarrow a = c - b$. Since $-b = b$, $a = b + c$.
\end{enumerate}

\subsection*{Set G}
\begin{enumerate}
    \item See Table \ref{tab:parity-c1}.
        \begin{table}[!hb]
            \centering
            \begin{tabular}{c|c|c}
                                        & $a_4 = a_1 + a_3$ & $a_5 = a_1 + a_2 + a_3$ \\ \hline
            $00000$ & $0 = 0 + 0$       & $0 = 0 + 0 + 0$         \\
            $00111$ & $1 = 0 + 1$       & $1 = 0 + 0 + 1$         \\
            $01001$ & $0 = 0 + 0$       & $1 = 0 + 1 + 0$         \\
            $01110$ & $1 = 0 + 1$       & $0 = 0 + 1 + 1$         \\
            $10011$ & $1 = 1 + 0$       & $1 = 1 + 0 + 0$         \\
            $10100$ & $0 = 1 + 1$       & $0 = 1 + 0 + 1$         \\
            $11010$ & $1 = 1 + 0$       & $0 = 1 + 1 + 0$         \\
            $11101$ & $0 = 1 + 1$       & $1 = 1 + 1 + 1$        
            \end{tabular}
            \caption{Parity-check equations for $C_1$}
            \label{tab:parity-c1}
        \end{table}
    \item $a_4 = a_2$, $a_5 = a_ + a_2$, $a_6 = a_1 + a_2 + a_3$, $a_i \in \mathbb{B}$.
        \begin{enumerate}
            \item $C_2 = \{000000, 001001, 010111, 011110, 100011, 101010, 110100, 111101\}$.
            \item Minimum distance: 2 (e.g., $000000$ and $001001$).
            \item There are $2^6 = 64$ words in $\mathbb{B}^6$ and there are $8$ codewords in $C_2$.
            To be detected, a codeword must be transformed in a non-codeword. So there are $64 - 8 = 36$ ways of doing that.
        \end{enumerate}
    \item $\{0000, 0101, 1011, 1110\}$, for equations $a_3 = a_1$ and $a_4 = a_1 + a_2$. Minimum distance: 2.
    \item Let $\dec$ be the decode function. So,
        $$\dec(11111) = 11101$$
        $$\dec(00101) = 00111$$
        $$\dec(11000) = 11010$$
        $$\dec(10011) = 10011$$
        $$\dec(10001) = 10011$$
        $$\dec(10111) = 10011, 00111$$
    \item If the minimum distance in a code is $m$, that means, by definition, that to transform one codeword into another, it is necessary to change at least $m$ bits.
        Therefore, if less than $m$ bits are changed, the result is a non-codeword and, as such, can be detected.
    \item Let us assume that there is a certain element $x \in \mathbb{B}: x \in S_t(a) \cap S_t(b)$.
        Then the largest possible value of $d(a, b)$ is $2t = m - 1$.
        But it takes at least $m$ errors to change one codeword into another. So, the premise is false and, therefore, $S_t(a) \cap S_t(b) \ne \emptyset$.
    \item Let us say a codeword $w$ is transformed into a non-codeword $w'$ such that $d(w, w') \leqslant t$. Then $w' \in S_t(w)$.
        Since $S_t(w) \cap S_t(x) = \emptyset$ for any other codeword $x$, $w'$ can be unambiguously decoded into $w$.
    \item \emph{I am probably wrong, but here is my reasoning, anyway}: the minimum distance in $C_1$ is 2. 
        If that is the case, ``two errors in any codeword can always be detected'' is false. 
        For instance, errors in positions 3 and 6 of $000000$ result in $001001$, another codeword, thus undetectable.
\end{enumerate}

\section*{Chapter 4}
\subsection*{Set A}
\begin{enumerate}
    \item $axb = c \Rightarrow aa^{-1}xb = a^{-1}c \Rightarrow xbb^{-1} = a^{-1}cb^{-1} \Rightarrow x = a^{-1}cb^{-1}$.
    \item $x^2b = xa^{-1}c \Rightarrow x^{-1}xxb = x^{-1}xa^{-1}c \Rightarrow xb = a^{-1}c \Rightarrow xbb^{-1} = a^{-1}cb^{-1} \Rightarrow x = a^{-1}cb^{-1}$.
    \item $x^2a = bxc^{-1} \Rightarrow x^2ac = bx$. But $xac = acx$, so $xacx = bx \Rightarrow xac = b \Rightarrow x = bc^{-1}a^{-1}$.
    \item $ax^2 = b \Rightarrow ax^3 = bx$. But $x^3 = e$, so $x = b^{-1}a$.
    \item $x^2 = a^2 \Rightarrow x^4 = a^4 \Rightarrow x^5 = a^4x$. But $x^5 = e$, so $e = a^4x \Rightarrow x = (a^4)^{-1}$.
    \item $(xax)^3 = bx \Rightarrow xax^2ax^2ax = bx$. But $x^2a = a^{-1}x^{-1}$, so $xaa^{-1}x^{-1}a^{-1}x^{-1}x = bx \Rightarrow a^{-1} = bx \Rightarrow x = (ab^{-1})$.
\end{enumerate}

\subsection*{Set B}
\begin{enumerate}
    \item False. $AA = I$, but $A \ne I$.
    \item False. $AA = I = II$, but $A \ne I$.
    \item False. $(AB)^2 = C^2 = I$, but $A^2B^2 = ID = D$.
    \item True. $x^2 = x \Rightarrow xxx^{-1} = xx^{-1} \Rightarrow x = e$.
    \item False. There is no $y$ such that $y^2 = A$.
    \item True. By the definition of groups, $x^{-1} \in G$. So $x^{-1}y = z \in G$ (groups are closed under the operation). Therefore $y = xz$. 
\end{enumerate}

\subsection*{Set C}
\begin{enumerate}
    \item $ab = ba \Rightarrow (ab)^{-1} = (ba)^{-1} \Rightarrow b^{-1}a^{-1} = a^{-1}b^{-1}$.
    \item $a = b^{-1}ba \Rightarrow a = b^{-1}ab \Rightarrow ab^{-1} = b^{-1}a$.
    \item $a(ab) = a(ba) = (ab)a$.
    \item $a^2b^2 = aabb = abab = baba = bbaa = b^2a^2$.
    \item $(xax^{-1})(xbx^{-1}) = xabx^{-1} = xbax^{-1} = (xbx^{-1})(xax^{-1})$.
    \item $aba^{-1} = baa^{-1} = b \Rightarrow aba^{-1}a = ba \Rightarrow ab = ba$.
    \item $e = ab(ab)^{-1} = ab(ba)^{-1} = aba^{-1}b^{-1}$.
\end{enumerate}

\subsection*{Set D}
\begin{enumerate}
    \item $ab = e \Rightarrow a = b^{-1} = ba = bb^{-1} = e$.
    \item $a(bc) = e \Rightarrow (bc)a = e$ (from item 1). Similarly, $(ab)c = e \Rightarrow cab = e$.
    \item If $a_1\ldots a_n = e$, then the product of all $a_i$, in any order, is equal to $e$.
    \item $xay = a^{-1} \Rightarrow xaya = e \Rightarrow yaxa = e \Rightarrow yax = e^{-1}$.
    \item $ab = c \Rightarrow abc = e \Rightarrow bca = e \Rightarrow bc = a$. Also, $cab = e \Rightarrow ca = b$.
    \item $abc = (abc)^{-1} \Rightarrow abcabc = e \Rightarrow bcabca = e \Rightarrow bca = (bca)^{-1}$. Also $cabcab = e \Rightarrow cab = (cab)^{-1}$.
    \item $abba = aea = aa = e \Rightarrow ab = (ba)^{-1}$.
    \item $a = a^{-1} \Rightarrow aa = a^{-1}a \Rightarrow aa = e$. Conversely, $aa = e \Rightarrow aaa^{-1} = ea^{-1} \Rightarrow a = a^{-1}$.
    \item $ab = c \Rightarrow abc = cc = e$. Conversely, $abc = e \Rightarrow abcc = ec \Rightarrow ab = c$.
\end{enumerate}.

\subsection*{Set E}
\begin{enumerate}
    \item Let us use the Algorithm \ref{alg:s-construction} to construct $S$.
        \begin{algorithm}[H]
            \caption{Construction of $S$}\label{alg:s-construction}
            \begin{algorithmic}[1]
                \Procedure{}{}
                \State $S \leftarrow \emptyset$
                \State $G' \leftarrow$ copy of $G$
                \While{$G'$ contains at least one element which is not its own inverse} 
                    \State{Add to $S$ one such element and its inverse}
                    \State{Remove the pair from $G'$}
                \EndWhile
                \EndProcedure
            \end{algorithmic}
        \end{algorithm}
        First of all, note that at each step, the algorithm removes two elements from $G'$. Since $G'$ is finite, the algorithm is guaranteed to terminate.
        At the end of each iteration, $S$ gains two new elements. So the property that $|S|$ is even is guaranteed throughout. Finally, when the algorithm stops,
        $G'$ does not contain any element that is its own inverse. Therefore, $S$ is complete.
    \item From item 1, we know that $|S| = 2k$, for some $k \in \mathbb{N}$. The number of elements that are equal to their own inverses is $|G| - |S|$. There are, then,
        two possiblities:
        \begin{equation*}
            |G| - |S| =
            \begin{cases}
                2(m - k)     & \text{if $G = 2m$, for some $m \geqslant k$}\\ 
                2(m - k) + 1 & \text{if $G = 2m + 1$, for some $m \geqslant k$}
            \end{cases}
        \end{equation*}
        Thus if the number of elements in $G$ is even, so is the number of elements in $G$ that are equal to their own inverses. Likewise, if $G$ has an ood number of elements.
    \item If $|G|$ is even, $|G| - |S| = 2m$, for some $m \in \mathbb{N}$. But $e$ is always its own inverse. So the number of elements $x \in G$ such that $x \ne e$
        and $x = x^{-1}$ is $2n + 1$, for some $0 \leqslant n < m$. So $2n + 1 \geqslant 1$.
    \item Let $|S| = k$. Then $G = \{a_1, \ldots, a_k, a_{k+1}, \ldots, a_n\}$. $G$ being abelian, we can rewrite $(a_1\ldots a_n)^2$ as 
        $$(a_1\ldots a_n)^2 = a_1a_1^{-1}\ldots a_ka_k^{-1}a_{k+1}a_{k+1}^{-1}\ldots a_{m}a_{m}^{-1} = e$$ where $m = \frac{n - k}{2}$.
    \item $a_1\ldots a_n = a_1a_1^{-1}\ldots a_{\frac{n}{2}}^{\phantom{-1}}a_{\frac{n}{2}}^{-1} = e$.
    \item Let's say, without loss of generality, that $a_1^{\phantom{1}} \ne a_1^{-1}$. Then $a_1\ldots a_n = a_1a_2a_2^{-1}\ldots a_{\frac{n-1}{2}}^{\phantom{-1}}a_{\frac{n-1}{2}}^{-1} = a_1$.
\end{enumerate}

\subsection*{Set F}
\begin{enumerate}
    \item 
        \begin{enumerate}
            \item $a^2 = a \Rightarrow aaa^{-1} = aa^{-1} \Rightarrow a = e$.
            \item $ab = a \Rightarrow a^{-1}ab = a^{-1}a \Rightarrow b = e$.
            \item $ab = b \Rightarrow abb^{-1} = bb^{-1} \Rightarrow a = e$.
        \end{enumerate}
    \item From exercise 4.B.6 we know that, for any two elements $x$ and $y$ in $G$ there is an element $z$ in $G$ such that $y = xz$. In table terms, this means
        that in each row, every element appears at least once. Now let us assume that for certain $x$, $y$ in $G$ there are $z_1$, $z_2$ in $G$ such that $y = xz_1 = xz_2$.
        Then $z_1 = z_2$. In table terms, this translates to each element in a row appearing in exactly one position.
    \item See Table \ref{tab:group-3}.
        \begin{table}[!hb]
            \centering
            \begin{tabular}{l|lll}
            &$e$&$a$&$b$\\ \hline
            $e$&$e$&$a$&$b$\\
            $a$&$a$&$b$&$e$\\
            $b$&$b$&$e$&$a$
            \end{tabular}
            \caption{Group with three elements}
            \label{tab:group-3}
        \end{table}
    \item See Table \ref{tab:group-4-self-inverses}.
        \begin{table}[!ht]
            \centering
            \begin{tabular}{l|llll}
            &$e$&$a$&$b$&$c$\\ \hline
            $e$&$e$&$a$&$b$&$c$\\
            $a$&$a$&$e$&$c$&$b$\\
            $b$&$b$&$c$&$e$&$a$\\
            $c$&$c$&$b$&$a$&$e$
            \end{tabular}
            \caption{Group with four elements such that $xx = e$ for every $x \in G$}
            \label{tab:group-4-self-inverses}
        \end{table}
    \item See Table \ref{tab:group-4-one-not-self-inverse}
        \begin{table}[]
            \centering
            \begin{tabular}{l|llll}
            &$e$&$a$&$b$&$c$\\ \hline
            $e$&$e$&$a$&$b$&$c$\\
            $a$&$a$&$b$&$c$&$e$\\
            $b$&$b$&$c$&$e$&$a$\\
            $c$&$c$&$e$&$a$&$b$
            \end{tabular}
            \caption{Group with four elements such that $xx = e$ for some $x \ne e \in G$ and $yy \ne e$ for some $y \in G$}
            \label{tab:group-4-one-not-self-inverse}
        \end{table}
    \item \textbf{TODO}.
\end{enumerate}

\subsection*{Set G}
\begin{enumerate}
    \item Prove that $G \times H$ is a group.
        \begin{enumerate}[label=(G\arabic*)]
            \item 
                $$(x_1, y_1)[(x_2y_2)(x_3y_3)] = (x_1, y_1)(x_2x_3, y_2y_3) = (x_1x_2x_3, y_1y_2y_3)$$
                $$[(x_1, y_1)(x_2y_2)](x_3y_3) = (x_1x_2, y_1y_2)(x_3, y_3) = (x_1x_2x_3, y_1y_2y_3)$$
            \item $e = (e_G, e_H)$, since $(x_1, y_1)(e_G, e_H) = (x_1, y_1) = (e_G, e_H)(x_1, y_1)$.
            \item $(x, y)^{-1} = (x^{-1}, y^{-1})$, since $(x, y)(x^{-1}, y^{-1}) = (e_G, e_H) = (x^{-1}, y^{-1})(x, y)$.
        \end{enumerate}
    \item $\mathbb{Z}_2 \times \mathbb{Z}_3 = \{(0, 0), (0, 1), (0, 2), (1, 0), (1, 1), (1, 2)\}$. See Table \ref{tab:z2xz3} for the group operation.
        \begin{table}[!hb]
            \centering
            \begin{tabular}{c|cccccc}
            $+$      & $(0, 0)$ & $(0, 1)$ & $(0, 2)$ & $(1, 0)$ & $(1, 1)$ & $(1, 2)$ \\ \hline
            $(0, 0)$ & $(0, 0)$ & $(0, 1)$ & $(0, 2)$ & $(1, 0)$ & $(1, 1)$ & $(1, 2)$ \\
            $(0, 1)$ & $(0, 1)$ & $(0, 2)$ & $(0, 0)$ & $(1, 1)$ & $(1, 2)$ & $(1, 0)$ \\
            $(0, 2)$ & $(0, 2)$ & $(0, 0)$ & $(0, 1)$ & $(0, 2)$ & $(0, 0)$ & $(1, 1)$ \\
            $(1, 0)$ & $(1, 0)$ & $(1, 1)$ & $(1, 2)$ & $(0, 0)$ & $(0, 1)$ & $(0, 2)$ \\
            $(1, 1)$ & $(1, 1)$ & $(1, 2)$ & $(1, 0)$ & $(0, 1)$ & $(0, 2)$ & $(0, 0)$ \\
            $(1, 2)$ & $(1, 2)$ & $(1, 0)$ & $(1, 1)$ & $(0, 2)$ & $(0, 0)$ & $(0, 1)$
            \end{tabular}
            \caption{Operation table for $\mathbb{Z}_2 \times \mathbb{Z}_3$}
            \label{tab:z2xz3}
        \end{table}
    \item $(g_1, h_1), (g_2, h_2) \in G \times H \Rightarrow (g_1, h_1)(g_2, h_2) = (g_1g_2, h_1h_2)$. Since $G$ and $H$ are abelian, 
        we can flip each multiplication in the tuple, resulting in $(g_2g_1, h_2h_1) =  (g1, h_1)(g_2, h_2)$.
    \item $(g, h)(g, h) = (gg, hh) = (e_G, e_H) = e_{G\times H}$.
\end{enumerate}

\subsection*{Set H}
\begin{enumerate}
    \item For $n = 1$: $(bab^{-1}) = ba^{-1}b^{-1}$. Now suppose $(bab^{-1})^{n} = ba^nb^{-1}$ for some $n \geqslant 1$. Then:
        $$(bab^{-1})^{n + 1} = (bab^{-1})^{n}(bab^{-1}) = ba^{n}b^{-1}bab^{-1} = ba^nab^{-1} = ba^{n + 1}b^{-1}$$
    \item For $n = 1$: $(ab)^1 = a^1b^1$. Now suppose $(ab)^n = a^nb^n$, for some $n \geqslant 1$. Then:
        $$(ab)^{n + 1} = (ab)^n(ab) = a^nb^nab = a^nab^nb = a^{n + 1}b^{n + 1}$$
    \item For $n = 1$: $(xa)^{2.1} = xaxa = ea = a = a^1$. Now suppose $(xa)^{2n} = a^n$ for some $n \geqslant 1$. Then:
        $$(xa)^{2(n + 1)} = (xa)^{2n + 2} = (xa)^{2n}xaxa = a^nea = a^{n + 1}$$
    \item $a^3 = a^2 = e \Rightarrow a^{-1} = a^2 \Rightarrow (a^{-1})^2 = a^3a = a$. So $\sqrt{a} = a^{-1}$.
    \item $a^2 = e \Rightarrow a^2a = ea \Rightarrow a^{-1} = a^2 \Rightarrow a^3 = a$. So $\sqrt[3]{a} = a$.
    \item If there is some $x$ such that $a^{-1} = x^3$, then $a = (a^{-1})^{-1} = (x^3)^{-1} = (xxx)^{-1} = x^{-1}x^{-1}x^{-1} = (x^{-1})^3$. Therefore, $\sqrt[3]{a} = x^{-1}$.
    \item 
    \item $xax = b \Rightarrow axax = ab \Rightarrow (ax)^2 = ab \Rightarrow \sqrt{ab} = ax$.
\end{enumerate}

\section*{Chapter 5}
\subsection*{Set A}
\begin{enumerate}
    \item $G = \langle \mathbb{R}, +, \rangle$, $H = \{\log a: a \in \mathbb{Q}, a > 0\}$. $H$ is a subgroup of $G$.
        \begin{enumerate}[label=(\roman*)]
            \item Suppose $\log a, \log b \in H$; then $\log a + \log b = \log(ab)$. Since $ab \in \mathbb{Q}$ and $ab > 0$, $\log(ab) \in H$. So $H$ is closed under addition.
            \item Suppose $\log a \in H$; then $-\log a = \log a^{-1} = \log \frac{1}{a}$. Since $ \frac{1}{a} \in \mathbb{Q}$ and $\frac{1}{a} > 0$, $-\log a \in H$.
        \end{enumerate}
    \item $G = \langle\mathbb{R}, +\rangle$, $H = \{\log n: n \in \mathbb{Z}, n > 0\}$. $H$ is \emph{not} a subgroup of $G$.
        \begin{enumerate}[label=(\roman*)]
            \item Suppose $\log m, \log n \in H$; then $\log m + \log n = \log(mn)$. Since $mn \in \mathbb{Z}$ and $mn > 0$, $\log(mn) \in H$.
            \item Suppose $\log n \in H$; then $-\log n = \log\frac{1}{n}$. But $\frac{1}{n} \notin \mathbb{Z}$. So $-\log n \notin H$.
        \end{enumerate}
    \item $G = \langle\mathbb{R}, +\rangle$, $H = \{x \in \mathbb{R}: \tan x \in \mathbb{Q}\}$. $H$ is a subgroup of $G$.
        \begin{enumerate}[label=(\roman*)]
            \item Suppose $x, y \in H$; then $\tan(x + y) = \frac{\tan x + \tan y}{1 - \tan x \tan y}$, which is rational. So $x + y \in H$.
            \item Suppose $x \in H$; then $\tan(-x) = -\tan x \in \mathbb{Q}$. So $-x \in H$.
        \end{enumerate}
    \item $G = \langle \mathbb{R}, \cdot \rangle$, $H = \{2^n3^m: m, n \in \mathbb{Z}\}$. $H$ is a subgroup of $G$.
        \begin{enumerate}[label=(\roman*)]
            \item Suppose $2^{n}3^{m}, 2^{p}3^{q} \in H$; then $2^{n}3^{m}2^{p}3^{q} = 2^{n + p}3^{m + q}$. Since $n + p, m + q \in \mathbb{Z}$,
                $H$ is closed under multiplication. 
            \item Suppose $2^{n}3^{m} \in H$; then $(2^{n}3^{m})^{-1} = 2^{-n}3^{-m}$. Since $-n, -m \in \mathbb{Z}$, $H$ is closed under inverses.
        \end{enumerate}
    \item $G = \langle \mathbb{R} \times \mathbb{R}, + \rangle$, $H = \{(x, y): y = 2x\}$. $H$ is a subgroup of $G$.
        \begin{enumerate}[label=(\roman*)]
            \item Suppose $(x_1, 2x_1), (x_2, 2x_2) \in H$; then $(x_1, 2x_1) + (x_2, 2x_2) = (x_1 + x_2, 2(x_1 + x_2))$. So, 
                $H$ is closed under addition. 
            \item Suppose $(x, 2x) \in H$; then $-(x, 2x) = (-x, -2x) = (-x, 2(-x))$. So, $H$ is closed under inverses.
        \end{enumerate}
    \item $G = \langle \mathbb{R} \times \mathbb{R}, + \rangle$, $H = \{(x, y): x^2 + y^2 > 0\}$. $H$ is \emph{not} a subgroup of $G$.
        \begin{enumerate}[label=(\roman*)]
            \item Suppose $(x, y) \in H$; then $(-x, -y)$ is also in $H$. But $(x, y) + (-x, -y) = (0, 0) \notin H$, since $0^2 + 0^2 = 0$. So, $H$ is not closed under addition. 
        \end{enumerate}
    \item \textbf{TODO}.
\end{enumerate}

\subsection{Set B}
\begin{enumerate}
    \item $G = \langle \mathscr{F}(\mathbb{R}), + \rangle$, $H = \{f \in \mathscr{F}(\mathbb{R}): f(x) = 0, \text{for every $x \in [0, 1]$}\}$. $H$ is a subgroup of $G$.
        \begin{enumerate}[label=(\roman*)]
            \item Suppose $f, g \in H$; then, for every $x \in [0, 1]$, $[f + g](x) = f(x) + g(x) = 0 + 0 = 0$. So, $f + g \in H$.
            \item Suppose $f \in H$; then, for every $x \in [0, 1]$, $[-f](x) = -f(x) = 0$. So, $-f \in H$.
        \end{enumerate}
\end{enumerate}

\end{document}
