\section{Chapter 9}
\subsection{Set A}
\begin{enumerate}
    \item
        \begin{enumerate}
            \item injective: $f(a) = f(b) \Rightarrow \epsilon(a) = \epsilon(b)$.
    
            \item surjective: $f(a) = a$, for every $a \in G$.
    
            \item $f(ab) = \epsilon(ab) = \epsilon(a)\epsilon(b) = f(a)f(b)$.
        \end{enumerate}
    
    \item
        \begin{enumerate}
            \item injective: $f^{-1}(a) = f^{-1}(b) \Rightarrow f(f^{-1}(a)) = f(f^{-1}(b)) \Rightarrow a = b$.
    
            \item surjective: $f^{-1}(f(a)) = a$, for any $a \in G_1$.
    
            \item $f^{-1}(f(a))f^{-1}(f(b)) = ab = f^{-1}(f(ab)) = f^{-1}(f(a)f(b))$, for any $a, b \in G_1$.
        \end{enumerate}
    item
        \begin{enumerate}
            \item injective: $ g(f(a)) = g(f(b)) \Rightarrow f(a) = f(b) \Rightarrow a = b $.
    
            \item surjective: $ g $ is surjective, so for any $ c \in G_3 $ there exists some $ b \in G_2 $ such that $ g(b) = c $. $ f $ is surjective, so for any $ b \in G_2 $ there exists some $ a \in G_1 $ such that $ f(a) = b $. Therefore, for any $ c \in G_3 $ there is some $ a \in G_1 $ such that $ g(f(a)) = c $.
    
            \item $ g(f(ab)) = g(f(a)f(b)) = g(f(a))g(f(b)) $.
        \end{enumerate}
\end{enumerate}

\subsection{Set B}
\begin{enumerate}
    \item For any $ a \in G_1 $, $ f(e_1)f(a) = f(e_1a) = f(a) $, Therefore, $ f(e_1) $ is the neutral element $ e_2 \in G_2 $.
    
    \item $ e_2 = f(e_1) = f(aa^{-1}) =f(a)f(a^{-1}) \Rightarrow f^{-1}(a) = f(a^{-1}) $.
    
    \item Any $ b \in G_1 $ can be written as $ b = a^n $, for some integer $n$. So every element of $ G_2 $ can be written as $ f(a^n) = f^n(a) $.
\end{enumerate}

\subsection{Set C}
\begin{enumerate}
    \item $ G \ncong H $. In $ G $, every element is its own inverse. In $ H $, $ i.i \ne 1 $.
    
    \item $ G \ncong H $. In $ \mathbb{Z}_4 $, $ 1 + 1 = 2 \ne 0 $.
    
    \item $ G \ncong H $. In $ G $, every element is its own inverse. In $ H $, $ i.i \ne 1 $.
    
    \item $ G \cong H $. $ A = (23) $, $ B = (123) $, $ C = (13) $, $ D = (132) $, $ K = (12) $, $ I = () $.
    
    \item \textbf{TODO}.
    
    \item $ G \ncong H $. In $ G $, every element is its own inverse. In $ H $, $ i.i \ne 1 $.
\end{enumerate}

\subsection{Set D}
\begin{enumerate}
    \item $ \mathbb{Z}_2 \times \mathbb{Z}_2 \cong P_2 $. $ f: P_2 \rightarrow \mathbb{Z}_2 \times \mathbb{Z}_2, f = 
    \begin{pmatrix}
        \emptyset & {a}    & {b}    & {a, b}\\
        (0, 0)    & (0, 1) & (1, 0) & (1, 1)
    \end{pmatrix} $.\\
    $ \mathbb{Z}_4 \cong V $. $ f: \mathbb{Z}_4 \rightarrow V, f = 
    \begin{pmatrix}
        0 & 1 &  2 & 3 \\
        1 & i & -1 & -i
    \end{pmatrix}$.
    
    \item $ \mathbb{Z}_3 \times \mathbb{Z}_2 \cong \mathbb{Z}_6 \cong \mathbb{Z}_7^* $. Each of these three groups is generated by a single element: $ (1, 1) $, $ 1 $ and $ 3 $, respectively. $ S_3 $ does not have this property, so it is not isomorphic to the other three groups.
    
    \item $ P_3 $.
    
    \item For this exercise, let's call the groups $ A $, $ B $, $ C $ and $ D $, respectively according to the order in which they appear in the book.

    $ A $ and $ B $ are generated by two elements each. In $ A $, let's call them $ a $ and $ b $. From the diagrams we can deduce that $ a^2 = b^3 = e $. In $ B $, let's call them $ c $ and $ d $. Similarly, $ c^2 = d^3 = e $. If they are to be isomorphic, a necessary condition is: any isomorphism between them maps $ a \to c $ and $ b \to d $. However, $ ba = ab^2 $ and $ dc = cd $. If they were isomorphic, $ dc $ would equal $ cd^2 $.

    $ C $ also has two generators, $ f $ and $ g $. But $ f^2 = g^2 = e $, which is different from $ A $ and $ B $. So $ C $ is not isomorphic to any of them.

    $ D $ has only one generator, so it's not isomorphic to any of the other groups.
\end{enumerate}

\subsection{Set E}
\begin{enumerate}
    \item $ f: \mathbb{Z} \to E, f(n) = 2n $.
        \begin{enumerate}
    
            \item injective: $ f(a) = f(b) \Rightarrow 2a = 2b \Rightarrow a = b $.
    
            \item surjective: by definition, for any $ m \in E $, $ m = 2k $, for some $ k \in \mathbb{Z} $. So $ f(k) = m $.
    
            \item $ f(a) + f(b) = 2a + 2b = 2(a + b) = f(a + b) $.
        \end{enumerate}
    
    \item $ f: \mathbb{Z} \to G, f(n) = 10^n $.
        \begin{enumerate}
    
            \item injective: $ f(a) = f(b) \Rightarrow 10^a = 10^b \Rightarrow a = b $.
    
            \item surjective: For any $ m \in G $, $ f(\log_{10} y) = y $.
    
            \item $ f(a)f(b) = 10^a.10^b = 10^{a + b} = f(a + b) $.
        \end{enumerate}
    
    \item $ f: \mathbb{R} \times \mathbb{R} \to \mathbb{C}, f(a, b) = a + bi $.
        \begin{enumerate}
            \item injective: $ f(a, b) = f(c, d) \Rightarrow a + bi = c + di \Rightarrow a = c, b = d \Rightarrow (a, b) = (c, d) $.

            \item surjective: by definition, for any $ x = a + bi \in \mathbb{C} $, $ f(a, b) = x $, for some pair $ (a, b) \in \mathbb{R} \times \mathbb{R}$.

            \item $ f(a, b) + f(c, d) = (a + bi) + (c + di) = (a + c) + (b + d)i = f(a + c, b + d) $.
        \end{enumerate}

    \item In $ \mathbb{R^*} $, $ -1.(-1) = 1 $, which is the identity element. In other words, $ -1 $ is a non-identity element that is its own inverse. In $ \mathbb{R} $, there is no such element, because $ x + x = 2x $. The only element that satisfies the equation $ 2x = 0 $ is $  x = 0 $. Therefore, $ \mathbb{R^*} \ncong \mathbb{R} $.

    \item $ \mathbb{Z} $ is generated by a single element, $ 1 $. There can be no single element that generates $ \mathbb{Q} $. Let's prove this by contradiction, by assuming there is such a generator $ \frac{a}{b} $. Then, for any other rational number, $ \frac{c}{d} $, there is an integer $ n $ such that $ \frac{a}{b}.n = \frac{c}{d} $. For two rationals to be considered equal, $ and = bc $. Since $ b \ne 0 $, we can rewrite this equation as $ c = \frac{d}{b}an $. That means that all rational numbers whose denominators are not multiples of $ b $ cannot be generated by $ \frac{a}{b} $. Therefore, there is no single generator for $ \mathbb{Q} $ and $ \mathbb{Z} \ncong \mathbb{Q} $. 

    \item Let's assume that there is an isomorphism, $ f: \mathbb{Q} \to \mathbb{Q}^{\mathbf{POS}} $. Being an isomorphism, there must exist some $ a \in \mathbb{Q} $ such that $ f(a) = 2 $. Also,
    $$
        f\left(\frac{a}{2} + \frac{a}{2}\right) = f\left(\frac{a}{2}\right)f\left(\frac{a}{2}\right) = \left[f\left(\frac{a}{2}\right)\right]^2 = 2
    $$

    But there is no element in $ \mathbb{Q}^{\mathbf{POS}} $ whose square is 2. Therefore, there is no isomorphism between $ \mathbb{Q}$ and $\mathbb{Q}^{\mathbf{POS}}$.
\end{enumerate}

\subsection{Set F}
\begin{enumerate}
    \item $ f: G_1 \to G_2 $ is an isomorphism in which $ f((24)) = a $, $ f((1234)) = b $:

    $ (24)^2 = (24)(24) = e $\\
    $ (1234)^4 = (1234)(1234)(1234)(1234) = e $\\
    $ (1234)(24) = (24)(1234)(1234)(1234) = (12)(34) $

    \item  $ f: G \to G' $ is an isomorphism in which $ f((23)) = a $ and $ f((13)) = b $.

    \item \textbf{TODO}

    \item \textbf{TODO}
\end{enumerate}

\subsection{Set G}
\begin{enumerate}
    \item 
        \begin{enumerate}
            \item injective: $ f(a) = f(b) \Rightarrow a - 1 = b - 1 \Rightarrow a = b $.

            \item surjective: For any $ y \in G $, $ f(y + 1) = y $.

            \item $ f(a) * f(b) = a - 1 + b - 1 + ab - a - b + 1 = ab -1 = f(ab) $.
        \end{enumerate}

    \item $ f: \mathbb{R} \to G, f(x) = x - 1 $. This function is bijective for the same reasons shown in the previous exercise. In addition, $ f(a) * f(b) = a - 1 + b - 1 + 1 = a + b - 1 = f(a + b) $.

    \item $ f(x) = 2x $.
        \begin{enumerate}
            \item injective: $ f(a) = f(b) \Rightarrow 2a = 2b \Rightarrow a = b $.

            \item surjective: $ f(y/2) = y$ for any $ y \in G $.

            \item $ f(a) * f(b) = \frac{2a2b}{2} = 2ab = f(ab) $.
        \end{enumerate}

    \item
        \begin{enumerate}
            \item injective: $ f(a, b) = f(c, d) \Rightarrow (-1)^ba = (-1)^dc $. Since $ b $ and $ c $ can only assume the values $ 0 $ and $ 1 $ and $ a $and $ c $ are both positive, we can conclude that $ (a, b) = (c, d) $.

            \item $ f\left(|y|, \frac{-y + |y|}{2|y|} \right) = y $ for any $y \in G$.

            \item $ f(a, b).f(c, d) = (-1)^ab.(-1)^cd = (-1)^{a + c}bd = f(bd, a + c) $. Obs.: in the case $  a = c = 1 $, in $ \mathbb{R}^* $, $ a + c = 2 $ and in $ \mathbb{Z}_2 $, $ a + c = 0 $. But in both cases, $ (-1)^{a + c} = 1 $, so the equality holds.
        \end{enumerate}
\end{enumerate}

\subsection{Set H}
\begin{enumerate}
    \item $ f: G \times H \to H \times G, f(a, b) = (b, a) $ is an isomorphism.
        \begin{enumerate}
            \item $ f(a, b) = f(c, d) \Rightarrow (b, a) = (d, c) \Rightarrow (a, b) = (c, d) $.

            \item surjective: $ f(b, a) = (a, b) $ for any $ (a, b) \in H \times G $.

            \item $ f(a, b)f(c, d) = (b, a)(d, c) = (bd, ac) = f(ac, bd) $.
        \end{enumerate}
    
    \item Let $ g: G_1 \to G_2 $ and $ h: H_1 \to H_2$ be isomophisms. Then $ f(a, b) = (g(a), h(b)) $.
        \begin{enumerate}
            \item injective: $ f(a, b) = f(c, d) \Rightarrow (g(a), h(b)) = (g(c), h(d)) \Rightarrow g(a) = g(c) $  and $ h(b) = h(d) $. Since $ g $ and $ h $ are isomophisms, $ a = c $ and $  b = d $. So $ (a, b) = (c, d) $.

            \item surjective: $ f(g^{-1}(a), h^{-1}(b)) = (a, b) $ for any $ (a, b) \in G _2 \times H_2 $.

            \item $ f(a, b)f(c, d) = (g(a), h(b))(g(c), h(d)) = (g(a)g(c), h(b)h(d)) $. Since $ g $ and $ h $ are isomophisms, $ g(a)g(c) = g(ac) $ and $ h(b)h(d) = h(bd) $. So, $ f(a, b)f(c, d) = (g(ac), h(bd)) = f(ac, bd) $.
        \end{enumerate}

    \item First, let's assume that $ f(x) = x^{-1} $ is an isomorphism from $ G $ to $ G $. Then, for any $ a, b \in G $:
    $$
        f(a)f(b) = f(ab) \Rightarrow a^{-1}b^{-1} = (ab)^{-1} \Rightarrow (ba)^{-1} = (ab)^{-1} \Rightarrow ba = ab
    $$

    Therefore, $ G $ is abelian. To prove the opposite direction, we can simply make the inferences above in the reversse order.

    \item $ f: G \to H, f(x) = x^{-1} $.
        \begin{enumerate}
            \item injective: $ f(a) = f(b) \Rightarrow a^{-1} = b^{-1} \Rightarrow a = b $.

            \item surjective: $ f(y^{-1}) = y $ for any $ y \in H $.

            \item $ f(a) * f(b) =a^{-1} * b^{-1} = b^{-1}a^{-1} = (ab)^{-1} $.
        \end{enumerate}
\end{enumerate}
    
\subsection{Set I}
\begin{enumerate}
    \item Since all elements of $ \mathbb{Z}_6 $ appear in the second row without repetition, $ f $ is bijective. $ f $ can be written as $ f(x) -x $. So,
    $$
        f(a) + f(b) = -a + (-b) = -(a + b) = f(a + b)
    $$

    \item $ f_3(x) = -x $. This is an automorphism for the same reason as the previous item.

    $ f_1(x) = 2x $. $ f_1(a) + f_1(b) = 2a + 2b = 2(a + b) = f(a + b) $.

    $ f_2(x) = 3x $. $ f_1(a) + f_1(b) = 3a + 3b = 3(a + b) = f(a + b) $.

    \item
        \begin{enumerate}
            \item injective: $ f(x) = f(y) \Rightarrow axa^{-1} = aya^{-1} \Rightarrow x = y $.

            \item $ f(a^{-1}ya) = y$ for any $ y \in G $.

            \item $ f(x)f(y) = axa^{-1}aya^{-1} = axya^{-1} = f(xy) $.
        \end{enumerate}

    \item
        For any $ z \in G $ there is a $ y \in G $ such that $ g(y) = z $ because $ g $ is an automorphism.\\
        For any $ y \in G $ there is a $ x \in G $ such that $ f(x) = y $ because $ f$ is an automorphism.\\
        Therefore, for any $ z \in G $ there is a $ x $ such that $ f(g(x)) = z $.

        So, $ \text{Aut}(G) $ is closed under composition. From exercise 9.A.2 we know that if $ f $ is an isomophism, then $ f^{-1} $ is also an isomophism. So $\text{Aut}(G)$ is also closed under inverses. Therefore, $\text{Aut}(G)$ is a subgroup of $ S_G $.
\end{enumerate}
